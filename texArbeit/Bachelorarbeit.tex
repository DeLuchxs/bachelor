% Statt "de" kann hier auch "en" stehen, wenn die Arbeit auf Englisch verfasst wird
% Entweder "darkstyle" oder "lightstyle"
\documentclass[de, darkstyle]{unirostock}
% Für die Quellenangaben, weitere Informationen https://de.overleaf.com/learn/latex/Bibliography_management_with_biblatex
\usepackage[backend=biber,style=alphabetic,maxbibnames=99,backref=true,citestyle=alphabetic]{biblatex}
\addbibresource{lit.bib}

% Welche Bedingungen gibt es für die Verbreitung der Arbeit?
% creative-commons: Creative Commons Lizenz, Namensnennung - Weitergabe erlaubt unter gleichen Bedingungen 
% private: Veröffentlichung und Veränderung nur nach Rücksprache mit dem Autor
\license{creative-commons}

\author{Max Mustermann}
\enrolmentnumber{123456789} % Matrikelnummer

\title{Hervorragende Untersuchungen mit interessanten Inhalten und bahnbrechenden Ideen}
\type{Masterarbeit}

\course{Informatik}
\workperiod{01. Mai 2019 -- 18. September 2019}
\supervisor{Dr.-Ing. Thomas Mundt}
\primaryreviewer{Prof. Dr. rer. nat. Clemens H. Cap}
\secondaryreviewer{Prof. Dr. rer. nat. habil. Andreas Heuer} % Falls es keinen gibt, einfach weglassen

\faculty{Fakultät für Informatik und Elektrotechnik}
\institute{Institut für Informatik}
\workinggroup{Lehrstuhl für Informations- und Kommunikationsdienste}

\begin{document}
\maketitle
\makelicense

\pagenumbering{Roman}

\tableofcontents % Inhaltsverzeichnis.
\clearpage

\setstretch{1.263}
\pagenumbering{gobble}
\clearpage
{\normalfont
\color{uniblau}
\huge\sffamily\itshape
Abstract
}

In dieser Machbarkeitsstudie wird untersucht, ob die Steuerung eines Schiffes durch Manipulation von Steuergeräten möglich ist.
Es wird ein Rogue Device entwickelt, das externe Steurungsbefehle erhält und manipulierte Nachrichten auf die Schiffskommunikation sendet.
Dabei liegt der Fokus der Arbeit auf der Motorsteuerung über einen CAN-Bus. Die Rudersteuerung mit einer seriellen Schnittstelle wird kurz betrachtet.
Eine zentrale Fragestellung ist dabei, wie die Manipulation von Steuergeräten erschwert werden kann.
Durch die Simulation eines Angriffs wird gezeigt, dass die Steuerung eines Schiffes durch Manipulation von Steuergeräten möglich ist.
Jedoch wird zusätzliche Forschung benötigt, um eine tatsächliche externe Steuerung zu ermöglichen.
Dadurch werden verschiedene Sicherheitslücken an dem CAN-Bus und dem J1939-Standard aufgezeigt.
Die Ergebnisse dieser Arbeit sind auch für andere Schiffe relevant, da der CAN-Bus weit verbreitet ist.

\vfill
\clearpage

\pagenumbering{arabic} % Ab hier folgt die "arabische" Seitennummerierung.

% Im Ordner chapter sollte für jedes Kapitel eine Datei angelegt und hier eingebunden werden.
% Das erhöht die Übersicht.
% Description: Einleitung der Arbeit
\chapter{Einleitung}
\printmyminitoc{1}

\section{Motivation}
In der Seefahrt ist die Anzahl an Cyberangriffen in den letzten Jahren stark angestiegen. Von 2021 auf 2022 hat sich die Anzahl von öffentlich
bekannten Angriffen mehr als verdoppelt. Im Jahr 2021 sind lediglich 24 Angriffe bekannt geworden, im Jahr 2022 waren es bereits 65 Angriffe 
\cite{mcad}. Darunter fallen Angriffe auf Häfen, Schifffahrtsunternehmen und Schiffe.
Jedes Jahr werden zahlreiche Passagiere mit Schiffen befördert. Dabei kann es vorkommen, dass bösartige Aktoren ein Teil dieser Passagiere sind.
Ein Angriff auf die Steuerung eines Schiffes kann katastrophale Folgen haben. Dennoch ist die Sicherheit von Schiffen nicht 
immer ausreichend gewährleistet \cite{Reilly2016}.
Um das zu verdeutlichen, soll in dieser Arbeit durch physischen Zugriff auf ein Schiff die Kommunikation manipuliert werden.
Damit soll die Kontrolle über die Steuerung des Schiffes übernommen werden. 
Auf großen Passagierschiffen ist das Risiko besonders hoch, da es
schwierig ist, einen solchen Angriff rechtzeitig zu bemerken. Es besteht die Gefahr, 
dass unbefugte Personen unbemerkten Zugriff zu kritischen Systemen erhalten. \\

\section{Ziel der Arbeit}
Diese Arbeit untersucht die Möglichkeit eines Angriffes auf die Steuerung eines gegebenes Beispielschiffes.
Dabei soll auf die notwendige Sicherheit in Schiffen aufmerksam gemacht werden.
Die Arbeit soll zeigen, wie wichtig es ist, die Systemkommunikation eines Schiffes zu schützen.
Durch die Untersuchung eines Angriffes sollen zwei zusätzliche Forschungsfragen beantwortet werden: \\
Wie kann die Manipulation von Steuergeräten auf Schiffen erschwert werden? \\
Wie relevant sind die Ergebnisse für andere Schiffe? \\
Dazu wird ein Rogue Device entwickelt, welches in der 
Lage, ist externe Steuerungsbefehle zu erhalten und mit diesen Kontrollnachrichten auf die Schiffskommunikation zu senden.
Dabei wird auf Netzwerke zugegriffen, die eine Kommunikation zwischen den verschiedenen Steuerungssystemen des Schiffes 
ermöglichen. Als Kommunikation spielt dabei der CAN-Bus eine wichtige Rolle.
Unter anderem wird auch 
auf eine serielle Schnittstelle zugegriffen, um mehr Kontrolle über das System zu erlangen. \\
Durch die Simulation eines Angriffes werden die Auswirkungen auf die Steuerung des 
Schiffes analysiert. Zusätzlich 
sollen auch mögliche Gegenmaßnahmen betrachtet werden. \\
Um die Sicherheitslücken zu veranschaulichen, soll das Schiff mit einem Gaming-Controller gesteuert werden.
Dies soll zeigen, wie einfach es ist, die Steuerung eines Schiffes zu übernehmen, wenn die Kommunikation nicht 
ausreichend geschützt ist.\\

\section{Aufbau der Arbeit}
Dazu werden zuerst die Grundlagen für die Arbeit erläutert. 
Um die Sicherheit von einem CAN-Bus zu diskutieren, muss dieser im Vorfeld verstanden werden. Daher wird
im Detail auf den CAN-Bus und den J1939-Standard eingegangen.
Im gleichen Kapitel wird auch auf den aktuellen Stand der Technik eingegangen.
Ein wichtiger Teil dabei sind schon vorhandene Tools, die als Hilfsmittel für die Arbeit genutzt werden können.
Im nächsten Schritt werden Konzepte für die Steuerung des Schiffes entwickelt. Dabei wird auf die Steuerungslogik eingegangen, aber
vor allem Integration des Rogue Devices in das System. Mit diesen Konzepten wird in die Implementierung eingestiegen. Dabei sind 
die Verbindung des Spiele-Controllers mit dem Rogue Device und die Kommunikation mit dem Schiff wichtige Schritte.
In der Kommunikation mit dem Schiff wird auf die CAN-Bus Kommunikation und die serielle Kommunikation eingegangen.
Damit kann die Machbarkeit des Angriffs getestet werden. Im Anschluss werden Sicherheitslücken aufgezeigt und mögliche Gegenmaßnahmen
vorgeschlagen. Zum Schluss wird die Relevanz der Ergebnisse für andere Schiffe betrachtet und ein Ausblick gegeben. \\


\clearpage
\listoffigures % Abbildungsverzeichnis
\printbibliography % Quellenverzeichnis
\clearpage
\pagenumbering{gobble}
\chapter*{Erklärung}
Hiermit erkläre ich, dass ich die vorliegende Masterarbeit selbständig verfasst und keine anderen als die angegebenen Quellen und Hilfsmittel benutzt habe.

Alle Stellen, die wörtlich oder sinngemäß aus Veröffentlichungen entnommen sind, sind als solche kenntlich gemacht.

Die Arbeit ist noch nicht veröffentlicht und ist in ähnlicher oder gleicher Weise noch nicht als Prüfungsleistung zur Anerkennung oder Bewertung vorgelegt worden.

Rostock, den \today
\\
 \\
\\
\begin{Form}
  \digsigfield{7cm}{3cm}{Unterschrift}
\end{Form}
\end{document}