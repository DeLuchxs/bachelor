% Statt "de" kann hier auch "en" stehen, wenn die Arbeit auf Englisch verfasst wird
% Entweder "darkstyle" oder "lightstyle"

\documentclass[de, darkstyle]{unirostock}
\usepackage{float}
\usepackage[bottom]{footmisc}
\usepackage{minted}
\usepackage{colortbl}
% Für die Quellenangaben, weitere Informationen https://de.overleaf.com/learn/latex/Bibliography_management_with_biblatex
\usepackage[backend=biber,style=alphabetic,maxbibnames=99,backref=true,citestyle=alphabetic]{biblatex}
\addbibresource{lit.bib}
\DeclareFieldFormat{titlecase}{#1} % Optional: Titelstil setzen
\defbibheading{bibliography}[]{%
  \chapter*{\color{uniblau}\huge\sffamily\normalfont \textit{Literatur}}%
}
% Welche Bedingungen gibt es für die Verbreitung der Arbeit?
% creative-commons: Creative Commons Lizenz, Namensnennung - Weitergabe erlaubt unter gleichen Bedingungen 
% private: Veröffentlichung und Veränderung nur nach Rücksprache mit dem Autor
\license{creative-commons}

\author{Jakob Engelbert Tomahogh}
\enrolmentnumber{221201101} % Matrikelnummer

\title{Sicherheitsanalyse durch Entwicklung eines Rogue Device zur Echtzeitmanipulation maritimer Steuerungssysteme}
\type{Bachelorarbeit}

\course{Informatik}
\workperiod{15. November 2024 -- 04. April 2025}
\primaryreviewer{M. Sc. Marvin Davieds}
\secondaryreviewer{Prof. Dr. rer. nat. Clemens H. Cap} % Falls es keinen gibt, einfach weglassen

\faculty{Fakultät für Informatik und Elektrotechnik}
\institute{Institut für Informatik}
\workinggroup{Lehrstuhl für Informations- und Kommunikationsdienste}

\begin{document}
\maketitle
\makelicense

\pagenumbering{Roman}

\tableofcontents % Inhaltsverzeichnis.
\clearpage

\setstretch{1.263}
\pagenumbering{gobble}
\clearpage
{\normalfont
\color{uniblau}
\huge\sffamily\itshape
Abstract
}

In dieser Machbarkeitsstudie wird untersucht, ob die Steuerung eines Schiffes durch Manipulation von Steuergeräten möglich ist.
Es wird ein Rogue Device entwickelt, das externe Steurungsbefehle erhält und manipulierte Nachrichten auf die Schiffskommunikation sendet.
Dabei liegt der Fokus der Arbeit auf der Motorsteuerung über einen CAN-Bus. Die Rudersteuerung mit einer seriellen Schnittstelle wird kurz betrachtet.
Eine zentrale Fragestellung ist dabei, wie die Manipulation von Steuergeräten erschwert werden kann.
Durch die Simulation eines Angriffs wird gezeigt, dass die Steuerung eines Schiffes durch Manipulation von Steuergeräten möglich ist.
Jedoch wird zusätzliche Forschung benötigt, um eine tatsächliche externe Steuerung zu ermöglichen.
Dadurch werden verschiedene Sicherheitslücken an dem CAN-Bus und dem J1939-Standard aufgezeigt.
Die Ergebnisse dieser Arbeit sind auch für andere Schiffe relevant, da der CAN-Bus weit verbreitet ist.

\vfill
\clearpage

\pagenumbering{arabic} % Ab hier folgt die "arabische" Seitennummerierung.

% Im Ordner chapter sollte für jedes Kapitel eine Datei angelegt und hier eingebunden werden.
% Das erhöht die Übersicht.
% Description: Einleitung der Arbeit
\chapter{Einleitung}
\printmyminitoc{1}

\section{Motivation}
Jedes Jahr werden zahlreiche Passagiere mit Schiffen befördert. Dabei kann es vorkommen, dass bösartige Personen ein Teil dieser Passagiere sind.
Ein Angriff auf die Steuerung eines Schiffes könnte katastrophale Folgen haben. Dennoch wird die Sicherheit von Schiffen oft vernachlässigt.
Wenn die Kommunikation auf einem Can-Bus nicht ausreichend geschützt ist, könnte ein Angreifer durch physischen Zugriff auf das Schiff die 
Kommunikation manipulieren und so die Steuerung des Schiffes übernehmen. Auf großen Passagierschiffen ist das Risiko besonders hoch, da es
schwierig ist, einen solchen Angriff zu bemerken.


\section{Ziel der Arbeit}
\begin{enumerate}
    \item Dadurch soll aufgezeigt werden, wie wichtig es ist, die Kommunikation auf einem CanBus zu schützen
    und Aufmerksamkeit auf die Sicherheit von Schiffen zu lenken.
\end{enumerate}

Die Arbeit erforscht die Möglichkeit eines solchen Angriffs. Dazu wird ein Rogue Device entwickelt, welches in der 
Lage ist externe Steuerungsbefehle zu erhalten und mit diesen Kontrollnachrichten auf die Schiffskommunikation zu senden.
Dabei wird auf einen Can-Bus zugegriffen, der die Kommunikation zwischen den verschiedenen Steuerungssystemen des Schiffes 
ermöglicht. Als Protokolle spielen dabei NMEA 2000, NMEA 0183 und J1939 eine wichtige Rolle. Unter anderem wird auch 
auf eine serielle Schnittstelle zugegriffen, um mehr Kontrolle über das System zu erlangen. \\
Hier soll die Machbarkeit eines solchen Angriffs gezeigt werden. Zusätzlich werden die Auswirkungen auf die Steuerung des 
Schiffes analysiert. Es soll ein Bewusstsein für die Sicherheit von Schiffen geschaffen werden und mögliche Gegenmaßnahmen
vorschlagen.

\clearpage
\chapter{Grundlagen}

\printmyminitoc{1}

\section{CAN-Bus}
Bei einem Can-Bus handelt es sich um eine serielle Netzwerktechnologie, 
welche mehrere Geräte mit einem Draht verbindet.
Die Entwicklung des Can-Bus wurde von Bosch im Jahr 1983 begonnen. 1986 wurde der erste Can-Bus Standard 
veröffentlicht.
Die Motivation der Entwicklung war die effiziente Kommunikation zwischen den Steuergeräten in einem Auto. 
Als günstige
Nebenwirkung konnte damit die Kabelmenge reduziert werden, da alle Geräte mit einem Bus verbunden werden können.
Durch die höhere Zuverlässigkeit und Funktionalität des Can-Bus, wurde dieser schnell in der Autoindustrie etabliert.
Aber auch in anderen Sektoren, wie z.B. der Medizintechnik, der Gebäudeautomation oder der Luftfahrt, spielt der
Can-Bus mittlwerweile eine wichtige Rolle.
\cite[Seiten 2-10]{Voss2008}
\\
Man sprich von einem Bussystem, da alle Geräte gleichberechtigt sind. Dass heißt, dass jedes Gerät Nachrichten 
senden und empfangen kann.
Die Nachrichten werden nach Broadcasting-Prinzip übertragen. Jede Nachricht wird von allen Knoten empfangen, 
aber nur die Knoten, die die Nachricht benötigen, verarbeiten sie. Diese werden nicht Einhaltung der Protokollregeln 
überprüft,
da dies zu einer größeren unnötigen Last auf dem Bus führen würde. 
Jedoch wird die Integrität der Nachrichten durch eine Prüfsumme sichergestellt. Das wird auch mit 
einer Acknowledge-Nachricht(ACK) bestätigt. Wird eine Nachricht nicht bestätigt, wird von dem Sender eine Fehlermeldung
auf den Bus gesendet.
Bei einer fehlerhaften Nachricht reagieren
die Knoten mit einer Fehlermeldung, die wieder der gesamte Bus empfängt. Wenn ein Knoten dauerhaft fehlerhafte
Nachrichten sendet, wird dieser vom Bus getrennt. Die auf dem Bus gesendeten Daten werden mit einer Nachrichten-ID
versehen, die die Priorität der Nachricht angibt. Die Nachrichten mit der niedrigsten ID haben die höchste Priorität.
Die Maximale Länge einer Nachricht beträgt 8 Byte. Durch die vergleichsweise geringe Länge der Nachrichten, kann 
eine geringe Latenz erreicht werden. Dabei kann eine höchste Baudrate von 1Mbit/s gesetzt werden.
\cite[Seiten 13-19]{Voss2008}
\\
Alle Knoten in einem Can-Bus sind mit einem Zweiadrigen Kabel verbunden. Diese werden als High(CAN\_H) und Low(CAN\_L) 
bezeichnet. Der Bus ist an beiden Enden mit einem Widerstand von 120 Ohm abgeschlossen um Reflexionen zu vermeiden.
\cite[Seite 132]{Voss2008}
\\


Data-Frames eines Can-Bus:
\begin{itemize}
    \item Start of Frame (SOF): Startbit
    \item Arbitration Field: Identifier
    \item Control Field: wird zur Datengröße und Nachrichtslänge verwendet
    \item Data Field: eigentliche Nutzdaten
    \item CRC Field: Prüfsumme
    \item End of Frame (EOF): Stopbit
\end{itemize}
\cite[Seite 36]{Voss2008}

Erweitertes Can-Protokoll: 
Standard Can-Protokoll hat 11 Bit Identifier, erweitertes Can-Protokoll hat 29 Bit Identifier.
Higher-Layer-Protokolle SAE J1939 wurde für die Kommunikation in Nutzfahrzeugen entwickelt.
SAE = Society of Automotive Engineers\\
Identifier wurde mit J1939 auf 29 Bit erweitert um mehr verschiedene Nachrichten zu ermöglichen.\\
Auf einem Can-Bus können der Standard 11 Bit Identifier und der erweiterte 29 Bit Identifier gleichzeitig verwendet werden.
Wenn zwei Nachrichten den gleichen 11 Bit Identifier haben, wird die Nachricht mit dem 11 Bit Identifier bevorzugt immer die
höhere Priorität haben. Die spezifierten Baudraten sind hier 250kbit/s und 500kbit/s. Die Hauptkomponenten des
erweiterten Identifier sind die Priorität, eine Parameter Group Number (PGN) und eine Quell-Adresse.
Das PGN-Feld ist aus 2 Bit Data Page, einer Protocol Data Unit (PDU) und einem PDU spezifischen Feld 
aufgebaut.
Jedes Gerät an dem Can-Bus muss eine valide Adresse und einen einzigartigen Namen haben. Vor 
der Kommunikation in einem Netzwerk muss ein Gerät sich einen Namen und eine Adresse geben.
\cite{Murvay2018}



\section{Raspberry Pi}
\begin{itemize}
    \item Ein Raspberry Pi ist ein Einplatinencomputer, der von der Raspberry Pi Foundation entwickelt wurde.
    \item Der Raspberry Pi ist ein vollwertiger Computer, der auf einem einzigen Board integriert ist. (vergleichsweise günstig, aber leistungsschwach)
    \item eignet sich gut um einfache Aufgaben zu erledigen, wie z.B. das Ausführen von Programmen, das Anzeigen von Informationen auf einem Bildschirm oder das Steuern von Geräten.
    \item Der benutzte Raspberry Pi 5 hat folgende Anschlüsse: 4 $\cdot$ USB, 2$\cdot$Micro-HDMI, 1$\cdot$Ethernet, 1$\cdot$USB-C für Stromversorgung, 1$\cdot$Micro-SD-Kartensteckplatz
\end{itemize}

\subsection{Raspberry Pi als Rogue Device}
Unter einem Rogue Device versteht man ein Gerät, welches sich unautorisiert und unauffällig in ein Netzwerk einbindet. \cite{Scarfone2008}
Dies kann ein Raspberry Pi sein, der sich in ein Netzwerk einbindet und Daten abgreift oder manipuliert. Hierüber können 
sich Angreifer Zugriff auf das Netzwerk verschaffen. 
\begin{itemize}
    \item eigener Code kann auf Rogue Device laufen
    \item kann von Angreifern von außen gesteuert werden
    \item kann auch genuzt werden um Informationen über das Netzwerk zu sammeln
\end{itemize}

\section{State of the Art}
\begin{itemize}
    \item Was gibt es schon für Lösungen?
    \item Was ist der aktuelle Stand der Technik?
    \item Was ist der aktuelle Stand der Forschung?
\end{itemize}
\subsection{Anbindung von Raspberry Pi an CanBus}
\begin{itemize}
    \item Raspbian OS hat seit 05.05.2015 eingebundenen Support für den Mikrochip MCP251x
\end{itemize}
\cite{Salunkhe2016}
\begin{itemize}
    \item PiCan2 ist ein CanBus Board für den Raspberry Pi
    \item HAT (Hardware Attached on Top) Standard
    \item erlaubt dem Raspberry Pi mit dem CanBus zu kommunizieren mit einer Geschwindigkeit von bis zu 1Mbit/s
    \item UCAN ist ein USB-CAN Adapter, der es ermöglicht ein beliebiges USB-Gerät mit dem CanBus zu verbinden
\end{itemize}
\cite{Pant2019}
\subsection{Übersetzung von CanBus Nachrichten}
\begin{itemize}
    \item J1939 (DBC-Dateien) (https://docs.fileformat.com/de/database/dbc/)
    \item NMEA 0183 
    \item NMEA 2000 $\rightarrow$ CanBoat (https://github.com/canboat/canboat)
    \item cantools Python Library 
\end{itemize}
\clearpage
\chapter{State of the Art}
\printmyminitoc{1}

Für eine Sicherheitsanalyse von maritimen Steuerungssystemen ist es wichtig, die aktuelle Technik und die
Möglichkeiten von Angreifern zu kennen. In diesem Kapitel wird der aktuelle Stand der Technik vorgestellt.
Der Fokus liegt dabei auf Technologien, die für die Arbeit relevant sind.

\section{Rogue Device}
Unter einem Rogue Device versteht man ein Gerät, das sich unautorisiert und unauffällig in ein Netzwerk integriert \cite{Scarfone2008}.
Dies kann ein Raspberry Pi sein, der sich in ein Netzwerk einbindet und Daten abfängt oder manipuliert. Hierüber können 
sich Angreifer Zugriff auf das Netzwerk verschaffen. 
Ein solches Gerät kann dazu verwendet werden, um eigenen Code auszuführen. Der kann beispielsweise Daten veränden oder 
weitere Angriffe vorbereiten. Zudem kann es von Angreifern aus der Ferne gesteuert werden, wodurch gezielte 
Manipulationen oder Spionageangriffe vereinfacht werden. Darüber hinaus lässt sich ein Rogue Device nutzen, um Informationen 
über das Netzwerk zu sammeln, etwa durch das Mithören von Kommunikation oder die Analyse von Sicherheitsmechanismen.
Das ermöglicht es, Schwachstellen im Netzwerk aufzudecken und gezielt auszunutzen.

\subsection{Angriff mit einem Rogue Device}
Die Idee einen Angriff mit einem Rogue Device durchzuführen, ist nicht neu. Im Jahr 2017 hat Buttigieg et al. \cite{Buttigieg2017}
den möglichen Angriff auf ein automobiles System mit einem Rogue Device untersucht. Bei dem System handelte es sich um einen
CAN-Bus, um das Armaturenbrett eines Fahrzeugs zu steuern. Der weitere Fahrzeugteil wurde in einer Simulation dargestellt.
Buttigieg et al. hat erforscht, ob ein Rogue Device als unautorisiertes 
Gerät in einem CAN-Bus eingebunden werden kann. 
Dazu wurden zuerst die Schwachstellen eines CAN-Bus untersucht. Diese
sind in Abschnitt \ref{sec:canBusVulnerabilities} näher beleuchtet. Es werden zwei verschiedene Angriffe beschrieben.
Der erste Angriff umfasst das Einfügen von bösartigem Code in ein originales Steuergerät. Die zweite Methode ist das Ersetzen
des originalen Steuergerätes durch ein Rogue Device. Die letztere Methode ist für diese Arbeit relevant, da dieser Ansatz
ähnlich ist, wie der in dieser Arbeit verfolgte. Allerdings wurde in der Arbeit von Buttigieg et al. nicht mit einem
J1939-Standard gearbeitet, sondern mit dem ursprünglichen CAN-Standard. Es sollte weiterhin ein Man-in-the-Middle-Angriff
mit dem Rogue Device realisiert werden. \\
Im Ergebnis konnte ein Rogue Device erfolgreich unentdeckt integiert werden. Es konnten manipulierte CAN-Nachrichten gesendet werden.
Die manipulierten Informationen wurden von den Instrumenten dargestellt. Jedoch wurde nicht die Möglichkeit untersucht, manipulierte
Nachrichten an das Motorsteuergerät zu senden.


\section{Arbeit mit dem CAN-Bus}
Die Technologie des CAN-Bus wird in vielen Bereichen eingesetzt.
Daher gibt es auch viele Tools und Bibliotheken, die es ermöglichen, mit dem CAN-Bus zu arbeiten.
In dem folgenden Abschnitt wird der aktuelle Stand der Technik betrachtet.

\subsection{Anbindung von Raspberry Pi an CAN-Bus} \label{sec:raspberryPiToCAN}	
In der folgenden Arbeit soll ein Raspberry Pi an einen CAN-Bus angeschlossen werden. 
Im Folgenden werden einige verschiede Möglichkeiten vorgestellt.\\
Die erste Möglichkeit ist die Verwendung eines Microchip MCP251x, der als CAN-Controller dient \cite{Salunkhe2016}. Dieser wird 
mit dem Microchip MCP2551 ergänzt, der als CAN-Transceiver dient. Diese Kombination ermöglicht es, den
Raspberry Pi mit dem CAN-Bus zu verbinden. Der MCP251x wird über die SPI-Schnittstelle (Serial Peripheral Interface) 
des Raspberry Pi angeschlossen. Dazu wird ein Treiber benötigt, der die Kommunikation zwischen dem Raspberry Pi und dem
MCP251x ermöglicht. Ein solcher Treiber ist bereits seit dem 05.05.2015 im Raspberry Pi OS (ehemals Raspbian OS)
integriert \cite{Salunkhe2016}. 
Um diese Möglichkeit zu vereinfachen, gibt es auch verschiedene HATs (Hardware Attached on Top), die solche Microchips
bereits integriert haben \cite{Pant2019}. Ein Beispiel dafür ist das PiCan2 von SK Pang Electronics. Dieses Board wird auf den 
GPIO-Pins (General Purpose Input/Output) des Raspberry Pi gesteckt. Es erlaubt dem Raspberry Pi mit dem CAN-Bus
zu kommunizieren. Das PiCan2 unterstützt eine Geschwindigkeit von bis zu 1Mbit/s. \\
Es besteht auch die Möglichkeit, einen USB-CAN-Adapter zu verwenden. Damit ist die Anbindung eines 
beliebigen Computer an den CAN-Bus möglich.
Ein Beispiel für einen solchen Adapter ist der UCAN von Fysetc \cite{FysetcUCAN}. 
Die Hardware und Firmware des Adapters
sind open-source. Die Verbindung zum Raspberry Pi erfolgt über USB-C.
Zum CAN-Bus müssen lediglich CAN\_H und CAN\_L und die Masse verbunden werden. Da CAN-Bus Treiber für Linux, Windows und Mac
vorhanden sind, kann der UCAN mit diesen Betriebssystemen verwendet werden. 



\subsection{Übersetzung von CAN-Nachrichten} \label{sec:canTranslation}
Viele Software Tools ermöglichen die Arbeit mit dem CAN-Bus. Diese Tools unterscheiden sich in ihrem Funktionsumfang
und den Anwendungsbereichen.
Einige Programme sind speziell für die Analyse und das Debugging von CAN-Nachrichten 
ausgelegt, während andere umfassende Entwicklungs- und Simulationsmöglichkeiten bieten. Im Folgenden werden die wichtigsten 
Eigenschaften, Einsatzbereiche und Besonderheiten der einzelnen Tools vorgestellt. \\
\begin{itemize}
    \item \textbf{canCommander}\footnote{\href{https://github.com/MatthewKuKanich/CAN_Commander}{CAN Commander} (besucht am 15.02.2025)}: 
    Dieses Tool ermöglicht das Senden und Empfangen von CAN-Nachrichten. Es bietet eine einfache 
    Benutzeroberfläche und umfangreiche Analysefunktionen. Mit canCommander können CAN-Nachrichten aufgezeichnet, 
    analysiert und bearbeitet werden. Dabei können auch Nachrichten gesendet und injiziert werden. Es wird auch eine
    schon vorbereitete Platine verkauft, die es ermöglicht, mit dem CAN-Bus zu arbeiten. Dennoch ist canCommander ein
    open-source Projekt und kann auch auf anderen Plattformen verwendet werden. Diese Plattformen beschränken sich aber
    auf Mikrocontroller, wie Arduino Uno oder 
    ESP32.
    \item \textbf{cantools}\footnote{\href{https://github.com/cantools/cantools/releases}{cantools} (besucht am 15.02.2025)}: 
    Diese Python-Bibliothek ermöglicht es, CAN-Nachrichten zu dekodieren. Mit cantools können 
    CAN-Daten in ein Menschenlesbares Format umgewandelt werden. Die Bibliothek 
    bietet eine Vielzahl von Funktionen und unterstützt verschiedene Protokolle und Datenformate. Damit ist cantools ist ein 
    nützliches Werkzeug für die Analyse und Verarbeitung von CAN-Nachrichten und eignet sich für die Entwicklung von 
    Anwendungen, die mit dem CAN-Bus arbeiten.
    \item \textbf{CANoe}\footnote{\href{https://www.vector.com/de/de/produkte/produkte-a-z/software/canoe/}{CANoe} (besucht am 15.02.2025)}: 
    Dieses eigenständige kommerzielle Tool wurde von der Firma Vector Informatik GmbH. entwickelt.
    Es bietet eine vielzahl an verschiedenen Funktionen. CANoe ermöglicht die Entwicklung, Simulation und Analyse von
    CAN-Netzwerken. Es unterstützt verschiedene Protokolle und 
    Datenformate.
    \item \textbf{can\_decoder}\footnote{\href{https://github.com/CSS-Electronics/can_decoder}{can\_decoder} (besucht am 15.02.2025)}: 
    Das Open-Source-Projekt ermöglicht das Dekodieren von CAN-Nachrichten. Es wurde von der Firma
    CSS-Electronics entwickelt und ist auf GitHub verfügbar. Allerdings ist das Projekt nicht mehr aktiv und wird nicht mehr
    weiterentwickelt. Trotzdem kann es ein nützliches Werkzeug für die Analyse und Verarbeitung von CAN-Nachrichten 
    sein.
    \item \textbf{SavvyCAN}\footnote{\href{https://savvycan.com/index.php}{SavvyCAN} (besucht am 15.02.2025)}: 
    Die C++ Anwendung kann CAN-Nachrichten aufzeichnen, welche zum Reverse-Engineering genutzt werden können.
    Hiermit können auch einzelne Nachrichten dekodiert werden. SavvyCAN ist ein Open-Source-Projekt und kann auf GitHub gefunden werden. 
    Zusätzlich gibt es Unterstützung auf Linux und Windows.
    \item \textbf{can-utils}\footnote{\href{https://github.com/linux-can/can-utils}{can-utils} (besucht am 15.02.2025)}: 
    Diese Set an Linux-Tools ermöglichen die Arbeit mit dem CAN-Bus. Damit ist es möglich, CAN-Nachrichten 
    zu senden und zu empfangen.
    Es ist ein Open-Source-Projekt und kann auf GitHub gefunden werden. Diese Tools bieten eine Grundlage für die Entwicklung von 
    Anwendungen, die mit dem CAN-Bus arbeiten.
\end{itemize}

\subsection{DBC-Dateien}
Es ist essentiell die Nutzlast der CAN-Bus Nachrichten Informationen zuzuordnen. 
Das kann in der Form einer DBC-Datei geschehen\cite{Choi2021}. 
Diese Datei schlüsselt auf, welche Informationen in den verschiedenen CAN-Nachrichten enthalten sind.
Da auch sensible Informationen in den Nachrichten enthalten sein können, wird diese Datei in den meisten 
Fällen nicht öffentlich verfügbar gemacht. Ohne diese Datei können zwar Brute-Force Ansätze verwendet werden,
allerdings ist dies sehr aufwendig und nicht zielführend. \\
Für die Verwendung der Tools, die in Abschnitt \ref{sec:canTranslation} genannt wurden, sind DBC-Dateien notwendig.
Diese werden genutzt, um die zu dekodierenden und kodierenden Nachrichten zu beschreiben. 
Bei \texttt{canCommander}
sind einige DBC-Dateien bereits vorinstalliert. Bei cantools kann die DBC-Datei in das Programm geladen werden.
\begin{figure}[H]
    \centering
    \includegraphics[scale=0.2]{images/CAN-DBC-File-Format-Explained-Intro-Basics_2.png}
    \caption{Auszug aus einer Beispiel DBC-Datei \cite{cssElectronics}}
    \label{fig:dbcfile}
\end{figure}
In Abbildung \ref{fig:dbcfile} ist ein Ausschnitt aus einer DBC-Datei zu sehen.
In dieser sind einzelne Nachrichten aufgelistet. Jede Nachricht hat eine ID, eine Länge und Signale. Diese Signale
haben eine Länge, einen Offset und einen Faktor. Die Signale sind die eigentlichen Informationen, die in den CAN-Nachrichten
kodiert sind. Sie stellen also die Nutzlast der Nachrichten dar. Mit diesen Informationen kann eine Nachricht erstellt werden.
Im Anschluss an die Nachrichtentypen werden in der DBC-Datei für bestimmte Signale die möglichen Eingaben definiert.
Das hilft bei der richtigen Wahl der Eingabe. 
\clearpage
\chapter{Konzept und Systemdesign}
\printmyminitoc{1}

\section{Aufbau Schiffsysteme}
Ein Schiff besteht aus vielen verschiedenen Systemen, welche verschiedene Aufgaben haben. In diesem Fall wird das
Steuerungssystem des Schiffes betrachtet. Dabei werden die wichtigen Systeme für die Arbeit abstrahiert, um die
Funktionsweise des Systems zu verstehen.
\begin{figure}[H]
    \centering
    \includegraphics[scale=0.3]{images/limandaSystem.png}
    \caption{Vereinfachte Darstellung der Systeme auf der Limanda}
\end{figure}
Die wichtigen Systeme für diese Arbeit begrenzen sich auf die Gashebel und den Autopiloten. Die beiden Gashebel sind jeweils
mit einem CAN-Bus verbunden. An diesen Bussen werden die Steuerbefehle für die Motoren gesendet. Dort sind unter anderem 
auch Bildschirme für den Schiffsführer angeschlossen. Als Higher-Layer-Protokoll wird J1939 genutzt. \\
Der Autopilot ist ein eigenständiges System, das über eine serielle Schnittstelle mit dem Ruder verbunden ist.
Dieser hat keine Verbindung zu den Gashebeln oder Motoren. Er kann lediglich Signale an die Rudersteuerung über
eine serielle Verbindung senden. Diese Verbindung ist unverschlüsselt.
Die Rudersteuerung ist mit dem Ruderstellmotor über eine unbekannte Verbindung angeschlossen. Diese Verbindung ist
jedoch nicht relevant für diese Arbeit.\\



\section{Steuerungslogik des Spiele-Controllers} \label{sec:steuerungslogik}
Der benutzte Spiele-Controller ist ein Xbox Series X Controller. Dieser wurde gewählt, da er eine gute Haptik hat und weit
verbreitet ist. Zusätzlich ist er kabellos und kann somit frei bewegt werden. Um die Steuerung des Schiffes zu
ermöglichen, müssen die Eingaben des Xbox-Controllers in Steuerbefehle umgewandelt werden. Dies passiert auf dem 
Raspberry Pi. Der Xbox-Controller wird über Bluetooth mit dem Raspberry Pi verbunden. Dort werden die Eingaben des Controllers
ausgelesen und in einem Python-Programm in Steuerbefehle umgewandelt. \\
Um eine einfache Steuerung zu ermöglichen, wird im folgenden die Tastenbelegung aufgeschlüsselt.
Um alle gewünschten Funktionen umzusetzen, werden nicht alle Tasten benötigt. 

\begin{figure}[H]
    \centering
    \includegraphics[scale=0.5]{images/vorderseite.jpg}
    \caption{Vorderseite des Xbox-Controllers \cite{XboxController}}
    \label{fig:vorderseite}
\end{figure}

\begin{figure}[H]
    \centering
    \includegraphics[scale=0.5]{images/rueckseite.jpg}
    \caption{Rückseite des Xbox-Controllers \cite{XboxController}}
    \label{fig:rueckseite}
\end{figure}

In den Abbildungen \ref{fig:vorderseite} und \ref{fig:rueckseite} sind die Tasten des Xbox-Controllers zu sehen.
Die Tastenbelegung ist wie folgt:

\begin{table}[H]
    \begin{tabular}{|c|c|}
    \hline
    \rowcolor[gray]{0.8}
     Nummerierung der Taste & Funktion \\ \hline 
     1 & Bewegung des Ruders \\ \hline 
     2 & Reduzierung der linken Gashebelposition \\ \hline 
     7 & Reduzierung der rechten Gashebelposition \\ \hline
     11 & Erhöhung der rechten Gashebelposition \\ \hline
     14 & Erhöhung der linken Gashebelposition \\ \hline
     B + 2 & Umschalten des Rückwärtsgangs am linken Motor \\ \hline
     B + 7 & Umschalten des Rückwärtsgangs am rechten Motor \\ \hline
     B + 2 + 7 & Umschalten des Rückwärtsgangs an beiden Motoren \\
      & (basierend auf dem derzeitigen Gang am rechten Motor) \\ \hline
    \end{tabular}
\end{table}
Das Einlegen des Rückwärtsgangs ist durch eine Tastenkombination so gewählt, dass es nicht aus Versehen passieren kann.
Mit jeweils der Taste 2 oder 7 wird die Gashebelposition reduziert. Mit der zusätzlichen Betätigung der Taste B wird der 
Rückwärtsgang eingelegt an dem jeweiligen Motor. Wenn die Tasten 2, 7 und B gleichzeitig betätigt werden, 
wird der Rückwärtsgang für beide Motoren gleichzeitig umgeschaltet.
Damit das Getriebe während des Schaltvorgangs keine Gaseingabe erhält und möglicherweise Schaden nimmt, 
ist eine Verzögerung von 10 Sekunden eingebaut. 
Das soll dem Getriebe genug Zeit geben, um den Gang zu wechseln.
Allerdings soll es auch eine Möglichkeit geben, beide Getriebe gleichzeitig umzuschalten, um nicht 
nacheinander die Verzögerung zu haben. Dies wird durch die Tastenkombination B + 2 + 7 realisiert.
Dabei wird der Gang des rechten Motors als Referenz genommen. \\

\section{Integration des Rogue Device}
Damit der Controller die Steuerbefehle an das Schiff senden kann, muss das Rogue Device in das System integriert werden.
In diesem Fall ist das Rogue Device der Raspberry Pi. Damit dieser möglichst unbemerkt in das System integriert werden kann,
muss der Controller drahtlos verbunden werden. Um eine unentdeckte Integration zu ermöglichen, muss die Art der Stromversorgung
unabhängig von dem Schiff sein. Dafür könnte ein Akku genutzt werden. Dieser müsste jedoch regelmäßig geladen werden.
Aber für eine einmalige Anwendung ist das ausreichend. \\
Um die Kommunikation von dem Rogue Device zu dem Schiff zu ermöglichen, müssen
die einzelnen Systeme angesteuert werden. Um die Gashebelposition zu verändern, wird der Raspberry Pi mit dem CAN-Bus des Schiffes
verbunden. \\
Die grobe Struktur des Rogue Devices soll wie folgt aussehen:
\begin{figure}[H]
    \centering
    \includegraphics[scale=0.4]{images/piKonzept.png}
    \caption{Konzept der Programmstruktur auf dem Rogue Device}
    \label{fig:structureRogueDevice}
\end{figure}
In \ref{fig:structureRogueDevice} wird der Controller über Bluetooth mit dem Raspberry Pi verbunden. Dieser liest die Eingaben des Controllers aus und
sendet die entsprechenden Steuerbefehle an das Schiff. Dafür wird der CAN-Bus des Schiffes genutzt. 
Sollte der Gashebel in der normalen Benutzung vom Schiffsführer benutzt werden, wird ein Signal an den CAN-Bus gesendet. 
Dieses Signal wird dann an die Motoren weitergeleitet. Dafür wird \texttt{canReader.py} genutzt. 
Wie in \ref{fig:structureRogueDevice} zu sehen, liest dieses Programm die
Nachrichten des CAN-Bus und reagiert auf die wichtigen Nachrichten.
Um echte Eingaben zu verhindern, muss auf diese Nachricht erkannt und darauf reagiert werden. 
In \texttt{canReader.py} wird die Nachricht gelesen und ein Signal für die Reaktion wird an \texttt{canInterpreter.py} 
gesendet. 
Dann kann eine Nachricht von dem Rogue Device gesendet werden, um die Gashebelposition
zu überschreiben. Eigene Nachrichten müssen dazu erkannt werden, um eine Endlosschleife von eigenen Nachrichten zu verhindern. Dafür könnte
der Nachrichtenzähler überwacht werden. Aus den aufgezeichneten Nachrichten hat sich herausgestellt, dass der 
Nachrichtenzähler wenige verschiedene Werte hat. Daher sind diese Werte zu den eigenen erstellten Nachrichten 
verschieden. Dies ist eine einfache Methode, um die eigenen Nachrichten zu erkennen. Eine bessere Möglichkeit ist es aber,
eine eigene CAN-ID zu berechnen. Diese kann dann genutzt werden, um die eigenen Nachrichten zu erkennen. \\
Wie in der Abbildung \ref{fig:j1939header} zu sehen, besteht der Header aus einer PGN, einer Quelladresse und einer Priorität. Um nun Nachrichten
an den Motor zu senden, kann eine Nachricht des Gashebels abgefangen werden. Die PGN kann für die eigene Nachricht genutzt
werden. Die Quelladresse kann auch kopiert werden. Die Priorität sollte möglichst klein gewählt werden, 
damit die Nachricht des Rogue Devices bevorzugt wird. In der eigenen Nachricht kann dann die gewünschte Gashebelposition gesendet werden.
\\
Um die Rudersteuerung zu manipulieren, muss der Raspberry Pi mit dem Autopiloten verbunden werden. Bei dem Autopiloten
handelt es sich um ein Navitron NT888G. Dieser ist über eine serielle Schnittstelle mit einer elektronischen Kontrolleinheit
für den Stellmotor des Ruders verbunden. Diese Verbindung ist unverschlüsselt, benutzt jedoch ein Propritäres Protokoll.
Durch reverse engineering müssen einzelne Befehle herausgefunden werden, welche für die Ruderkontrolle von Bedeutung sind.
Dazu müssen die Nachrichten des Autopiloten abgefangen und analysiert werden. Mit den gewonnenen Informationen können
eigene Nachrichten erstellt werden oder bestehende Nachrichten in eigener Reihenfolge abgespielt werden.
Zusätzlich kann das Ruder auch physikalisch über das Steuerrad bewegt werden. Dieser Weg wird jedoch nicht betrachtet.
Ein weiterer Ansatz wäre es, die echten Kursdaten abzufangen und zu manipulieren. Dadurch müsste der Autopilot reagieren
und das Ruder steuern. Die Kursdaten werden nach dem NMEA-0183-Standard gesendet. Dabei handelt es sich um ein Schnittstellenstandard, 
wobei die Daten in ASCII-Form gesendet werden. Der Standard ist für die Einweg-Kommunikation von einem Sender zu einem oder mehreren
Empfängern ausgelegt. Es gibt einige Erklärungen der Nachrichten, jedoch ist die offizielle Spezifikation kostenpflichtig. 
\cite{nmea0183} (letzter Zugriff: 07.03.2025)\\
Trotzdem sind für alte Systeme genug Informationen verfügbar, um die Nachrichten zu verstehen. Damit können die Kursdaten
einfach verstanden und manipuliert werden. Mit dem Austauschen der echten Daten durch manipulierte, wäre ein 
physikalischer Man-in-the-Middle Angriff realisiert.\\

\subsection{Rückmeldung der Eingaben}
Es muss eine Art der Rückmeldung geben, um in etwa die Eingaben im Vergleich zum momentanen Zustand zu sehen.
Dabei sollte die Rückmeldung möglichst unauffällig sein. Ein kleiner Bildschirm könnte benutzt werden, allerdings
muss dieser physisch an den Raspberry Pi angeschlossen werden. Das würde das Verstecken des Rogue Devices erschweren.
Eine App auf einem Handy könnte die Rückmeldung geben. Diese App könnte dann die gewünschten Positionen anzeigen.
Dafür muss der Raspberry Pi mit dem Handy verbunden werden. Dies könnte über Bluetooth geschehen.
Hier ist zu beachten, dass die Verbindung stabil sein muss und nicht von anderen Geräten gestört wird.
Jedoch kann ist hier mit einem größeren Aufwand zu rechnen, da die App erst entwickelt werden muss.
Eine weitere Möglichkeit würden Vibrationen im Xbox Controller sein. Diese könnten genutzt werden, wenn die derzeitige
Eingabe ein Maximum oder Minimum erreicht hat. Allerdings ist dies nicht so genau wie eine Anzeige auf einem Bildschirm.
Es könnten auch nur wenige bestimmte Positionen mittgeteilt werden. \\
Eine solche Rückmeldung ist wichtig, um eine sogenannte Pilot Induced Oscillation (PIO) zu verhindern. Das Phänomen tritt
vorallem bei Flugzeugen auf, wenn der Pilot zu stark gegensteuert \cite{McRuer1995}. Dabei reagiert das Flugzeug auf die Steuerbefehle
des Piloten und der Pilot reagiert auf die Reaktion des Flugzeugs. Dies führt zu einer Schwingung, die sich immer weiter
verstärkt. \\
Ein vergleichbares Phänomen kann auch bei Schiffen auftreten, wenn auch nicht im gleichen Ausmaß. Bei Schiffen 
sind die Bewegungen im Wasser langsamer. Dennoch kann es zu einer Schwingung kommen, wenn der Schiffsführer
zu stark gegensteuert. Um dies zu verhindern, ist eine Rückmeldung der Eingaben wichtig. Diese Rückmeldung kann dann
dazu genutzt werden, um die Eingaben genauer zu wählen.
\clearpage
\chapter{Implementierung}
\printmyminitoc{1}

\section{Verbindung Rogue Device - Controller}
Benutzte Hardware, Protokolle, Libraries

\section{Übersetzung Signale Controller - Schiff}
Welches Dateiformat wird für Controllersignale benutzt?
Wie werden diese effizient genug in Motorsignale übersetzt?
Kann ich einfach originale Steuerungssignale unterdrücken?

\section{Eingabe-Interface}
Wie wird die Rückmeldung tatsächlich aussehen?
\clearpage
\chapter{Sicherheit}
\printmyminitoc{1}

\section{Schwachstellen}
Die durchgeführte Arbeit konnte in diesem Format nur durch einige Schwachstellen in der Kommunikation durchgeführt werden.
Um einen solchen Angriff möglichst zu vermeiden, ist es wichtig, dass die Schwachstellen aufgedeckt und behoben werden.
Dafür werden in dem folgenden Abschnitt die aufgedeckten Schwachstellen dieser Arbeit aufgeführt und diskutiert.

\subsection{Schwachstellen am CAN-Bus}
Der CAN-Bus ist eines der zentralen Systeme in modernen Fahrzeugen und Schiffen. So auch auf der Limanda.
Nach dem CAN-Standard sind die Nachrichten auf dem Bus nicht verschlüsselt. Zusätzlich werden die Nachrichten nur 
auf ihre Richtigkeit überprüft, jedoch nicht auf ihre Authentizität. Das bedeutet, dass ein Angreifer Nachrichten auf den Bus
senden kann, die von anderen Systemen als legitim angesehen werden.\cite{Voss2008} (Seiten 13-19) Durch das Bus-Netzwerk wird die einfache Integration weiterer
Geräte in das System ermöglicht. Ein Angreifer kann so durch physischen Zugriff auf den Bus ein Gerät hinzufügen, welches im schlimmsten Fall
die Kontrolle über das gesamte System übernehmen kann. Die einzige Hürde, ein Gerät in ein CAN-Bus Netzwerk zu integrieren, ist das
Beschaffen der richtigen DBC-Datei. Diese Datei wird häufig von den Herstellern der Geräte nicht öffentlich zur Verfügung gestellt.
Es sind jedoch einige DBC-Dateien im Internet verfügbar, die von anderen Nutzern erstellt wurden. Durch das Ausprobieren von dem Dekodieren
der CAN-Nachrichten mit verschiedenen DBC-Dateien, kann eine passende Datei gefunden werden. \\
In dieser Arbeit wurde zuerst ausgenutzt, dass die Nachrichten durch physischen Zugriff auf den Bus mitgelesen werden können.
Die Nachrichten wurden aufgezeichnet und mit einer passenden DBC-Datei dekodiert. Nachdem die Nachrichten verstanden wurden, konnte
eine Nachricht mit dem gleichen Syntax und eigenen Werten erstellt werden. Die aufgezeichnete Nachrichten-ID wurde dabei in der Quelle 
verändert. Die Priorität ist gleich geblieben, da es bereits die höchste Priorität war. In anderen Fällen ist es möglich, die Priorität
einfach zu verändern. Mit der eigenen Nachrichten-ID konnten nun die 
echten Nachrichten von den eigenen unterschieden werden. Das ist wichtig, um echte Nachrichten mit den eigenen zu überschreiben. 
Alle Geräte in einem CAN-Bus sind gleichberechtigt\cite{Voss2008} (Seite 14). Dadurch wird die Überschreibung der älteren Nachrichten möglich.\\
Die vorher beschriebenen Schwachstellen sind allgemein für CAN-Bus Netzwerke. Wobei nicht alle eine Schwachstelle beschreiben, sondern 
Eigenschaften des Standards sind, welche zusammen mit den Schwachstellen ausgenutzt werden können.\\

\subsection{Schwachstellen am J1939-Protokoll}
Das J1939-Protokoll ist ein Protokoll, welches nicht für die Sicherheit, sondern für die Kommunikation zwischen den Steuergeräten
entwickelt wurde. Es ist ein Protokoll, welches auf dem CAN-Bus aufbaut. Das Protokoll ist nicht verschlüsselt und die Nachrichten
sind nicht authentifiziert. 
Allerdings hat das Protokoll auch spezifische Schwachstellen, welche aber nicht alle in dieser Arbeit ausgenutzt wurden.\\
Eine Schwachstelle ist, dass Nachrichten mit der niedrigsten ID die höchste Priorität haben. Zusätzlich ist es möglich, 
eine eigene Priorität so zu setzen, dass die Nachrichten mit der eigenen ID die höchste Priorität haben. Durch das Senden von
vielen Nachrichten mit hoher Priorität kann ein Denial-of-Service-Angriff durchgeführt werden. Dabei wird der Bus so stark belastet,
dass die echten Nachrichten nicht in einem angemessenen Zeitraum gesendet werden können. Jedes Gerät in einem J1939-Netzwerk
benötigt eine eindeutige Adresse und einen einzigartigen Namen \cite{JungerJ1939}. Da ein Gerät vollständige Kontrolle über 
den eigenen Namen und die eigene Adresse hat, kann ein Angreifer sich als ein anderes Gerät ausgeben. Die Namensübernahme führt
zunächst zu einem Konflikt, da zwei Geräte den gleichen Namen haben. Diesen Konflikt gewinnt das Gerät, welches die niedrigere
Adresse hat. Daher kann der Name recht einfach übernommen werden. Dadurch kann ein originelles Gerät aus dem Netzwerk ausgeschlossen
werden. Mit der übernommenen Adresse kann der Angreifer nun Nachrichten senden, die von anderen Geräten als legitim angesehen werden.
Die einzige Sicherheitsmaßnahme, die das J1939-Protokoll bietet, ist eine Zuweisungstabelle für die Adressen und Namen. Allerdings
bietet diese bei der Namensübernahme keinen Schutz, wenn der Angreifer auch die Adresse übernimmt. Diese Schwachstelle 
wurde in dieser Arbeit nicht ausgenutzt, da es zu unvorhersehbaren Folgen hätte führen können. Das spricht für die 
Schwere der Schwachstelle.\\
Eine weitere Möglichkeit den Bus zu stören, ist die globale Anfrage nach PGNs. Bei einer solchen Anfrage sollen alle Geräte
mit der eigenen PGN antworten. Die Spezifikation rät dazu, maximal 3 Anfragen pro Sekunde für eine Parametergruppe zu senden.
Es gibt aber keine feste obere Grenze für die Anzahl der Anfragen. Daher gibt es auch keine Gegenmaßnahmen bei zu vielen 
Anfragen. Ein Angreifer kann also durch das Senden von vielen Anfragen alle Geräte auf dem Bus dazu bringen, zu antworten.
Damit ist ein Distributed-Denial-of-Service-Angriff (DDoS) möglich, weil die Überlastung von mehreren Geräten ausgeht. \\
\cite{Murvay2018}


\section{Schutzmaßnahmen}
Welche gibt es bereits?
\begin{itemize}
    \item richtige Baudrate muss eingestellt sein (keine wirkliche Schutzmaßnahme)
    \item keine dedizierten Schutzmaßnahmen auf dem Forschungsschiff
\end{itemize}
Was sind weitere Möglichkeiten?
\begin{itemize}
    \item Verschlüsselung der Kommunikation
    \item Authentifizierung der Nachrichten
    \item Überwachung der Kommunikation
    \item regelmäßige Überprüfung der Kommunikation
    \item Auf größeren Schiffen: Trennung der Passagiernetzwerke von den Steuerungssystemen
\end{itemize}

\section{Relevanz für andere Schiffe}
Gibt es solche Angriffsmöglichkeiten auch auf anderen Schiffen?
\begin{itemize}
    \item Can-Bus häufig genutzt, häufig unverschlüsselt
    \item besonders auf größeren Schiffen ist das Ruder nicht mehr mechanisch, sondern elektronisch
    \item Angriff auf das Steuerungssystem könnte katastrophale Folgen haben
\end{itemize}
\clearpage
\chapter{Abschließende Betrachtung}
\printmyminitoc{1}

\section{Ergebnis}
\begin{itemize}
    \item CAN-Bus Nachrichten aufgezeichnet und in Echtzeit dekodiert
    \item Manipulierte CAN-Bus Nachrichten gesendet \begin{itemize}
        \item dabei keine Fehlermeldung im Steuergerät
        \item Motordrehzahl schießt in die Höhe (verhält sich unerwartet)
        \item kann zu Schäden am Motor führen
    \end{itemize}
    \item Programm kann mit canReader.py auf Nachrichten des Gashebels reagieren
\end{itemize}

\section{Ausblick}
\begin{itemize}
    \item Motordrehzahl konnte bisher nicht auf vorhersehbare Weise manipuliert werden
    \item aufgezeichnete Nachrichten konnten nicht alle dekodiert werden
    \item darunter auch Nachrichten für die Gangschaltung
    \item Manipulation von Nachrichten für die Gangschaltung
    \item Rudersteuerung ist nicht implementiert
    \item Nachrichten dafür nicht ganz verstanden
    \item tatsächliche Implementierung von vorgeschlagenen Sicherheitsmaßnahmen
    \item Prüfen auf die Effektivität der Sicherheitsmaßnahmen
\end{itemize}
\clearpage
% Römische Kapielnummern für Anhang
\renewcommand{\thechapter}{\Roman{chapter}}
\chapter{Anhang}
\printmyminitoc{1}

\section{Quellcode}

\section{Schaltpläne}

\section{Bildverzeichnis}
\listoffigures % Abbildungsverzeichnis

\section{Literaturverzeichnis}
\printbibliography % Quellenverzeichnis
\clearpage

\phantomsection
\addcontentsline{toc}{chapter}{Abbildungen}
\listoffigures
\clearpage
\phantomsection
\markboth{\uppercase{Literatur}}{}
\addcontentsline{toc}{chapter}{Literatur}
\printbibliography

\input{chapter/erklaerung}
\end{document}