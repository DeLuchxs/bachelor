\chapter{Sicherheit}
\printmyminitoc{1}

\section{Schwachstellen}
Welche habe ich benutzt und welche weiteren möglichen Schwachstellen habe ich gefunden?
\begin{itemize}
    \item Kommunikation auf dem Can-Bus ist nicht verschlüsselt
    \item Keine Authentifizierung der Nachrichten
    \item Einfach, weiteres Gerät in das System zu integrieren
    \item Serielle Schnittstelle am Autopiloten unverschlüsselt
    \item lediglich physischer Zugriff notwendig
    \item Daten in bestimmten Format, allerdings mit recht wenig Aufwand zu verstehen
\end{itemize}
\section{Schutzmaßnahmen}
Welche gibt es bereits?
\begin{itemize}
    \item richtige Baudrate muss eingestellt sein
    \item keine dedizierten Schutzmaßnahmen auf dem Forschungsschiff
\end{itemize}
Was sind weitere Möglichkeiten?
\begin{itemize}
    \item Verschlüsselung der Kommunikation
    \item Authentifizierung der Nachrichten
    \item Überwachung der Kommunikation
    \item regelmäßige Überprüfung der Kommunikation
    \item Auf größeren Schiffen: Trennung der Passagiernetzwerke von den Steuerungssystemen
\end{itemize}

\section{Relevanz für andere Schiffe}
Gibt es solche Angriffsmöglichkeiten auch auf anderen Schiffen?
\begin{itemize}
    \item Can-Bus häufig genutzt, häufig unverschlüsselt
    \item besonders auf größeren Schiffen ist das Ruder nicht mehr mechanisch, sondern elektronisch
    \item Angriff auf das Steuerungssystem könnte katastrophale Folgen haben
\end{itemize}