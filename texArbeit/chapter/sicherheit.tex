\chapter{Sicherheit}
\printmyminitoc{1}

\section{Schwachstellen}
Welche habe ich benutzt und welche weiteren möglichen Schwachstellen habe ich gefunden?
\begin{itemize}
    \item Kommunikation auf dem Can-Bus ist nicht verschlüsselt
    \item Keine Authentifizierung der Nachrichten
    \item Einfach, weiteres Gerät in das System zu integrieren
    \item Serielle Schnittstelle am Autopiloten unverschlüsselt
    \item lediglich physischer Zugriff notwendig
    \item Daten in bestimmten Format, allerdings mit recht wenig Aufwand zu verstehen
\end{itemize}
Spezifische Schwachstellen für J1939:
\begin{itemize}
    \item Nachrichten mit der niedrigsten ID haben die höchste Priorität, damit können DoS Angriffe durch das Senden von Nachrichten mit niedriger ID durchgeführt werden
    \item Es ist möglich, ein Node in das Netzwerk hinzuzufügen, welcher seinen NAME, Adresse ändern kann und vollständige Kontrolle über alles veränderbaren Felder hat (CANoe)
    \item Angreifer-Node kann später hinzugefügt werden und dann eine schon vorhandene Adresse übernehmen, durch ein NAME-field mit niedrigerem Wert und somit höherer Priorität
    \item ständig diesen ehrlichen Knoten zu stören eine Adresse zu erlangen kann diesen komplett aus der Kommunikation ausschließen
\end{itemize}
Sicherheitsmaßnahme von J1939:
Es gibt eine Zuweisungstabelle für Adresse-zu-NAME correspondence, welche jedoch einfach umgangen werden
kann, durch das deklarieren des gleichen NAME wie der gezielte Knoten, jedoch mit einer niedrigeren ID
um eine höhere Priorität zu erhalten und die Adresszuweisung zu gewinnen.
\begin{itemize}
    \item Angreifer kann sich als ein anderer Knoten ausgeben und so falsche, bösartige Nachrichten senden
    \item bei einer globalen Anfrage nach PGNs sollte jeder Knoten antworten
    \begin{itemize}
        \item Spezifikation rät dazu, maximal 3 Anfragen pro Sekunde für eine Parametergruppe
        \item keine Gegenmaßnahmen gegen mehr Anfragen
        \item somit kann ein DDoS Angriff durchgeführt werden (alle Knoten antworten bei einer Anfrage)
    \end{itemize}
\end{itemize}

\cite{Murvay2018}

\section{Schutzmaßnahmen}
Welche gibt es bereits?
\begin{itemize}
    \item richtige Baudrate muss eingestellt sein
    \item keine dedizierten Schutzmaßnahmen auf dem Forschungsschiff
\end{itemize}
Was sind weitere Möglichkeiten?
\begin{itemize}
    \item Verschlüsselung der Kommunikation
    \item Authentifizierung der Nachrichten
    \item Überwachung der Kommunikation
    \item regelmäßige Überprüfung der Kommunikation
    \item Auf größeren Schiffen: Trennung der Passagiernetzwerke von den Steuerungssystemen
\end{itemize}

\section{Relevanz für andere Schiffe}
Gibt es solche Angriffsmöglichkeiten auch auf anderen Schiffen?
\begin{itemize}
    \item Can-Bus häufig genutzt, häufig unverschlüsselt
    \item besonders auf größeren Schiffen ist das Ruder nicht mehr mechanisch, sondern elektronisch
    \item Angriff auf das Steuerungssystem könnte katastrophale Folgen haben
\end{itemize}