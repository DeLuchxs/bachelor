\clearpage
{\normalfont
\color{uniblau}
\huge\sffamily\itshape
Abstract
}

In dieser Machbarkeitsstudie wird untersucht, ob die Steuerung eines Schiffes durch Manipulation von Steuergeräten möglich ist.
Es wird ein Rogue Device entwickelt, das externe Steurungsbefehle erhält und manipulierte Nachrichten auf die Schiffskommunikation sendet.
Dabei liegt der Fokus der Arbeit auf der Motorsteuerung über einen CAN-Bus. Die Rudersteuerung mit einer seriellen Schnittstelle wird kurz betrachtet.
Eine zentrale Fragestellung ist dabei, wie die Manipulation von Steuergeräten erschwert werden kann.
Durch die Simulation eines Angriffs wird gezeigt, dass die Steuerung eines Schiffes durch Manipulation von Steuergeräten möglich ist.
Jedoch wird zusätzliche Forschung benötigt, um eine tatsächliche externe Steuerung zu ermöglichen.
Dadurch werden verschiedene Sicherheitslücken an dem CAN-Bus und dem J1939-Standard aufgezeigt.
Die Ergebnisse dieser Arbeit sind auch für andere Schiffe relevant, da der CAN-Bus weit verbreitet ist.

\vfill