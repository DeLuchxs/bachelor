% Description: Einleitung der Arbeit
\chapter{Einleitung}
\printmyminitoc{1}

\section{Motivation}
Jedes Jahr werden zahlreiche Passagiere mit Schiffen befördert. Dabei kann es vorkommen, dass bösartige Personen ein Teil dieser Passagiere sind.
Ein Angriff auf die Steuerung eines Schiffes könnte katastrophale Folgen haben. Dennoch wird die Sicherheit von Schiffen oft vernachlässigt.
Wenn die Kommunikation auf einem Can-Bus nicht ausreichend geschützt ist, könnte ein Angreifer durch physischen Zugriff auf das Schiff die 
Kommunikation manipulieren und so die Steuerung des Schiffes übernehmen. Auf großen Passagierschiffen ist das Risiko besonders hoch, da es
schwierig ist, einen solchen Angriff zu bemerken.


\section{Ziel der Arbeit}
\begin{enumerate}
    \item Dadurch soll aufgezeigt werden, wie wichtig es ist, die Kommunikation auf einem CanBus zu schützen
    und Aufmerksamkeit auf die Sicherheit von Schiffen zu lenken.
\end{enumerate}

Die Arbeit erforscht die Möglichkeit eines solchen Angriffs. Dazu wird ein Rogue Device entwickelt, welches in der 
Lage ist externe Steuerungsbefehle zu erhalten und mit diesen Kontrollnachrichten auf die Schiffskommunikation zu senden.
Dabei wird auf einen Can-Bus zugegriffen, der die Kommunikation zwischen den verschiedenen Steuerungssystemen des Schiffes 
ermöglicht. Als Protokolle spielen dabei NMEA 2000, NMEA 0183 und J1939 eine wichtige Rolle. Unter anderem wird auch 
auf eine serielle Schnittstelle zugegriffen, um mehr Kontrolle über das System zu erlangen. \\
Hier soll die Machbarkeit eines solchen Angriffs gezeigt werden. Zusätzlich werden die Auswirkungen auf die Steuerung des 
Schiffes analysiert. Es soll ein Bewusstsein für die Sicherheit von Schiffen geschaffen werden und mögliche Gegenmaßnahmen
vorschlagen.
