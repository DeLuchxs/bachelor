\chapter{Evaluation}
\printmyminitoc{1}

Mit den fertiggestellten Implementierungen und Sicherheitsbetrachtungen werden in diesem Kapitel die
Versuche betrachtet und die Ergebnisse evaluiert. Es werden sowohl erfolgreich umgesetzte Aspekte als auch
Herausforderungen betrachtet und analysiert.

\section{Manipulation der Motorsteuerung} \label{sec:evManipulationMotor}
Im ersten Test wurden die Nachrichten auf dem CAN-Bus des Schiffes im normalen Betrieb mit dem Programm \texttt{canReader.py} 
dekodiert. Dazu wurde ein Laptop mit einem Linux Betriebssystem verwendet. Dieser wurde über ein USB-C-Kabel mit einem 
UCAN-Adapter verbunden. Dieser UCAN-Adapter wurde bereits in Abschnitt \ref{sec:raspberryPiToCAN} beschrieben. 
Dabei hat die Echtzeitdekodierung der Nachrichten funktioniert. Es konnten die wichtigen Nachrichten für die Motorsteuerung
erkannt werden. Allerdings konnten nicht alle Nachrichten dekodiert. Zusätzlich wurden keine Nachrichten für die Gangschaltung
dekodiert. Durch das Handbuch
des Motors konnte eine Nachricht nachgestellt werden. Eine solche Nachricht wurde aber nicht in der Kommunikation des 
Schiffes gefunden. Daher wurde auch kein Versuch mit den nachgestellten Nachrichten durchgeführt. Damit hat sich 
das weitere Vorgehen auf die Drehzahl des Motors fokussiert. \\
In Kapitel \ref{sec:canBus} wurde die Implementierung der CAN-Bus Nachrichtenkodierung beschrieben. 
Damit konnte ein Versuch
aufgebaut werden, um den tatsächlichen Effekt der Nachrichten zu testen, die im Rahmen der Arbeit manipuliert wurden. 
Für den zweiten Versuch wurde wieder ein Laptop mit einem Linux Betriebssystem verwendet, anstatt des Raspberry Pi. 
Das hat den Vorteil, dass
die Programme einfacher gestartet und debuggt werden können. Der Laptop wurde mit einem UCAN-Adapter verbunden.  
Dieser wurde wiederum mit dem CAN-Bus des Schiffes verbunden. Um ein leichteres Debuggen zu ermöglichen, wurde bei diesem 
Versuchsaufbau kein Spiele-Controller verwendet. Stattdessen wurde die gleiche Nachricht mehrmals gesendet. Die Zieldrehzahl in 
dieser Nachricht beträgt 650 Umdrehungen pro Minute. Gleichzeitig wurde das Siagnal \texttt{'EngRequestedTorque\_TorqueLimit'} 
auf 0\% gesetzt. In der Vorbereitung wurde die Gangschaltung von dem Kapitän in den neutralen
Gang geschaltet. \\
Der Motor wurde gestartet und mit dem
Senden dieser Nachricht konnte kein Fehler auf dem CAN-Bus festgestellt werden. Gleichzeitig stieg die Drehzahl des Motors
schnell an. Trotz der Zielvorgabe von 650 RPM stieg die Drehzahl auf 1500 RPM. Bei dieser Drehzahl wurde der
Motor abgeschalten und die Nachrichtenübertragung gestoppt. Das Motorsteuergerät hat sich nicht nach den Erwartungen verhalten,
die aus dem J1939-Standard abgeleitet wurden. Um mögliche Schäden am Motor zu vermeiden, wurde von weiteren Tests abgesehen.
Daher konnte auch nicht die Reaktion auf Gashebelbewegungen getestet werden. \\
Die Ergebnisse des Versuchs zeigen, dass die Manipulation von CAN-Bus Nachrichten auf einem Schiff möglich ist.
Obwohl die manipulierten Nachrichten nicht den gewünschten Effekt hatten, konnte die Drehzahl des Motors erhöht werden.
Dadurch konnte ein Sicherheitsrisiko aufgezeigt werden. 

\section{Manipulation der Rudersteuerung}
Die Manipulation der Rudersteuerung wurde lediglich in einem Konzept in Abschnitt \ref{sec:manipulationRuder} beschrieben.
Eine Implementierung dieser Manipulation wurde nicht umgesetzt. 
Der verfolgte Ansatz für die Manipulation war, die Nachrichten zwischen der Eingabe des Autopiloten und der 
Verteilungseinheit zu manipulieren. Wie in Abbildung \ref{fig:limandaSystem} zu sehen ist, handelt es sich hierbei um eine 
serielle Verbindung.
Dabei sollte der Kurs der originalen Nachrichten durch die manipulierten Nachrichten
ersetzt werden. Das Problem bei der Implementierung war, dass die Nachrichten nicht dekodiert werden konnten. 
Das Format der Nachrichten konnte nicht endgültig verstanden werden. Es handelt sich um ein proprietäres Format,
welches keine öffentliche Dokumentation hat. \\
Aufgrund der zeitlichen Begrenzung der Arbeit, konnte der zweite Ansatz nicht weiter verfolgt werden. 
Der zweite Ansatz war die Manipulation der NMEA-0183 Nachrichten, die von den Sensoren gesendet werden.
Ein Nachteil dieses Ansatzes ist, dass mehr Verbindungen manipuliert werden müssen.
Durch die Nutzung von NMEA-0183 besteht ein möglicher Angriffspunkt für die Manipulation der Rudersteuerung.
Daher kann das als potentielles Sicherheitsrisiko angesehen werden.