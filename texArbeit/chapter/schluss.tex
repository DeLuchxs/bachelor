\chapter{Abschließende Betrachtung}
\printmyminitoc{1}

\section{Ergebnis}
In dieser Arbeit sollte die Machbarkeit von Angriffen auf das Steuergerät eines Schiffes untersucht werden. 
Der Fokus lag dabei auf dem CAN-Bus für die Motorsteuerung und einer seriellen Verbindung für die Rudersteuerung.
Um Schwachstellen zu veranschaulichen, sollte der Motor sowie das Ruder mittels eines Spiele-Controllers gesteuert werden.
Im ersten Schritt wurde ein Konzept für die Steuerungslogik entwickelt, wie in \ref{sec:steuerungslogik} zu sehen.
Damit die physichen Eingaben genutzt werden konnten, mussten diese in logische Eingabewerte für ein Programm umgewandelt werden.
Das wurde in dem Programm \texttt{controllerInput.py} umgesetzt (\ref{sec:signalControllerSchiff}). 
Damit auf eigenen Systemen CAN-Nachrichten getestet werden konnten, wurde ein CAN-Bus aufgebaut und Testnachrichten gesendet.
Im nächsten Schritt wurden aufgezeichnete CAN-Bus Nachrichten mithilfe einer DBC-Datei dekodiert und anschließend analysiert. 
Dabei wurden auch Nachrichten für die Rudersteuerung aufgezeichnet, diese konnten jedoch nicht dekodiert werden.
Durch die Analyse der Nachrichten, konnten einzelne wichtige Nachrichten entdeckt werden. Durch die weitergehende Analyse
der entsprechenden DBC-Datei zu den bestimmten Nachrichten, konnten diese auch verstanden werden (\ref{sec:canBus}).
Damit konnte ein Programm entwickelt werden, welches die Eingaben des Spiele-Controllers erhält und in CAN-Bus Nachrichten umwandelt.
Dazu musste auch eine entsprechende Prüfsumme berechnet werden, um die Nachrichten zu validieren. Das gleiche Vorgehen wurde auch
für Nachrichten der Gangschaltung durchgeführt. Ein weiterer wichtiger Teil war die Echtzeitdekodierung der CAN-Bus Nachrichten,
damit auf Eingaben des Gashebels reagiert werden konnte. Dazu wurde das Programm \texttt{canReader.py} entwickelt. 
Die Rudersteuerung wurde nicht implementiert, da die Nachrichten nicht im Zeitrahmen dekodiert werden konnten.
\\
Im ersten Test wurden die Nachrichten auf dem CAN-Bus des Schiffes im normalen Betrieb mit dem Programm \texttt{canReader.py} dekodiert.
Dabei hat die Echtzeitdekodierung der Nachrichten funktioniert. Es konnten die wichtigen Nachrichten für die Motorsteuerung
erkannt werden. Allerdings konnten nicht alle Nachrichten dekodiert. Zusätzlich wurden keine Nachrichten für die Gangschaltung
dekodiert. Dabei ist wichtig zu erwähnen, dass diese Nachrichten nicht in der Standard DBC-Datei enthalten waren. Durch das Handbuch
des Motors konnte eine Nachricht nachgestellt werden. Eine solche Nachricht wurde aber nicht in der Kommunikation des Schiffes gefunden.
Im zweiten Test wurden manipulierte Nachrichten auf den CAN-Bus gesendet. In der Vorbereitung wurde die Gangschaltung von dem Kapitän in den neutralen
Gang geschaltet. In diesem Zustand wurden Nachrichten mit einer Zieldrehzahl von 650 Umdrehungen pro Minute gesendet. Dabei ist die 
Drehzahl zügig angestiegen, aber über 650 Umdrehungen pro Minute hinaus. Bei einer Drehzahl von 1500 Umdrehungen pro Minute wurde die
Nachrichtenübertragung gestoppt, um mögliche Schäden am Motor zu vermeiden. Dabei ist zu erwähnen, dass keine Fehlermeldung im Steuergerät 
aufgetreten ist. Das Verhalten des Motorsteuergeräts war nicht erwartbar nach den Nachrichten aus dem J1939-Standard. Daher
konnte auch nicht die Reaktion auf Gashebelbewegungen getestet werden. 
\subsection{Wie kann die Manipulation von Steuergeräten erschwert werden?}
\begin{itemize}
    \item Sicherheitsmaßnahmen wurden vorgeschlagen (welche)
    \item Aufmerksamkeit von Schiffsführern für Gefahren sensibilisieren und richtige Verhalten bei Angriffen
\end{itemize}
\subsection{Relevanz für andere Schiffe}
Der Mehrwert dieser Arbeit fokussiert sich nicht nur auf die Limanda, sondern auch auf andere Schiffe.
Aufgrund der weiten Verbreitung des CAN-Bus, ist davon auszugehen, dass auch viele andere Schiffe diesen nutzen.
Der CAN-Bus der Limanda hat keine speziellen Sicherheitsmaßnahmen, die über den Standard hinausgehen. 
Die Limanda wurde erst 2021 in Betrieb genommen und ist daher auf einem aktuellen Stand der Technik \cite{limanda} (letzter Zugriff 14.03.2025).
Es kann davon ausgegangen werden, dass auch andere Schiffe, die in den letzten Jahren in Betrieb genommen wurden, auf einem
ähnlichen Stand der Technik sind. Daher ist es wahrscheinlich, dass der CAN-Bus auch auf anderen Schiffen keine speziellen
Sicherheitsmaßnahmen hat. \\
In dem Fall der Limanda wurde nur der Motor mit einem CAN-Bus angesteuert. Es besteht jedoch die Möglichkeit, dass auch 
andere Systeme, wie die Rudersteuerung, mit einem CAN-Bus angesteuert werden. Auf größeren Schiffen kann das eher der Fall sein. 
Eine mechanische Rudersteuerung ist auf größeren Schiffen nicht mehr üblich, da die Brücke zu weit von dem Ruder entfernt ist.
Daher kann die Rudersteuerung elektronisch sein. \\
In dieser Arbeit wurde gezeigt, dass NMEA-0183 Nachrichten einfach mitgelesen werden können. Dadurch können mindestens 
alle Informationen, die über NMEA-0183 übertragen werden, abgefangen werden. Es gibt bereits einen Nachfolger von NMEA-0183,
NMEA-2000. der Standard ist auch auf der Limanda implementiert, allerdings war dieser nicht relevant für diese Arbeit.
Deswegen kann keine Einschätzung über die Sicherheit von NMEA-2000 gemacht werden. \\
In dieser Arbeit konnten die Motoren nur manipuliert werden, wenn diese schon in Betrieb waren. Zusätzlich konnte
Manipulation durch das ausschalten der Motoren verhindert werden. Im Rahmen dieser Arbeit kann keine Aussage über 
die Möglichkeit der Manipulation von Motoren gemacht werden, die nicht in Betrieb sind. Es würde eine größere Gefahr
darstellen, wenn die Motoren von einem Angreifer ein- und ausgeschaltet werden könnten. \\

\section{Ausblick}
In dieser Arbeit wurde gezeigt, dass die Steuerung eines Schiffsmotor durch Manipulation von CAN-Bus Nachrichten möglich ist.
Allerdings konnte die Manipulation nicht auf vorhersehbare Weise durchgeführt werden. Die Motordrehzahl konnte nicht
vorhersehbar manipuliert werden. Eine Lösung dafür könnte in anderen Nachrichten von dem CAN-Bus liegen, welche nicht
dekodiert werden konnten. Daher könnte die Dekodierung von weiteren Nachrichten untersucht werden.
In den dekodierten Nachrichten konnten keine Informationen über die Gangschaltung gefunden werden. Daher ist anzunehmen,
dass diese Nachrichten auch nicht dekodiert werden konnten. Mit einer vollständigen Dekodierung der aller Nachrichten,
ist es wahrscheinlich, die Gangschaltung zu manipulieren. \\
Zusätzlich konnte nicht gezeigt werden, dass die Rudersteuerung manipuliert werden kann. Dies ist der begrenzten Zeit
und den fehlenden Informationen über die Rudersteuerung geschuldet. Es ist daher empfehlenswert, die Rudersteuerung
in einer weiteren Arbeit zu untersuchen. Dazu sollten die Nachrichten für die Rudersteuerung dekodiert werden. \\
In dieser Arbeit wurden Sicherheitsmaßnahmen vorgeschlagen, die nicht implementiert wurden. Diese müssen in einer weiteren
Arbeit auf Effektivität geprüft werden. Dazu können die Sicherheitsmaßnahmen implementiert und getestet werden.
Dabei kann auch die Reaktion auf Angriffe getestet werden. \\