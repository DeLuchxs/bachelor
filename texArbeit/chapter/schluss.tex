\chapter{Abschließende Betrachtung}
\printmyminitoc{1}

Mit den fertiggestellten Implementierungen und Sicherheitsbetrachtungen werden in diesem Kapitel die 
Ergebnisse evaluiert. Dabei werden die aufgestellten Forschungsfragen beantwortet und ein Ausblick auf
weitere Forschungsmöglichkeiten gegeben.

\section{Ergebnis}
In dieser Arbeit sollte die Machbarkeit von Angriffen auf das Steuergerät eines Schiffes untersucht werden, mit dem Ziel, 
das Schiff fernzusteuern.
Der Fokus lag dabei auf dem CAN-Bus für die Motorsteuerung und der Rudersteuerung. Um das Ruder zu steuern lag der Fokus
vor allem auf einer Manipulation an dem Autopiloten.
Damit Schwachstellen zu veranschaulicht werden, sollte der Motor sowie das Ruder mittels eines Spiele-Controllers gesteuert werden. \\
Zunächst wurde in Abschnitt \ref{sec:steuerungslogik} ein Konzept für die Steuerungslogik entwickelt.
Damit die physichen Eingaben des Spiele-Controllers für die Motor- und Rudersteuerung gentutzt werden können, 
mussten diese in logische Eingabewerte für ein Programm umgewandelt werden.
Das wurde in dem Programm \texttt{controllerInput.py} umgesetzt (Abschnitt \ref{sec:signalControllerSchiff}). 
Damit auf eigenen Systemen CAN-Nachrichten getestet werden konnten, wurde ein CAN-Bus aufgebaut und Testnachrichten gesendet.
Im nächsten Schritt wurden aufgezeichnete CAN-Bus Nachrichten mithilfe einer zur Verfügung gestellten 
DBC-Datei dekodiert und anschließend analysiert. 
Dabei wurden auch Nachrichten für die Rudersteuerung aufgezeichnet, diese konnten jedoch nicht dekodiert werden.
Durch die Analyse der CAN-Nachrichten, konnten einzelne wichtige Nachrichten entdeckt werden. Durch die weitergehende Analyse
der entsprechenden DBC-Datei zu den bestimmten Nachrichten, konnten diese auch verstanden werden (Abschnitt \ref{sec:canBus}).
Damit konnte das Programm \texttt{canInterpreter.py} entwickelt werden, welches die Eingaben des Spiele-Controllers erhält und 
in CAN-Bus Nachrichten umwandelt.
Dazu musste auch eine entsprechende Prüfsumme berechnet werden, um die Nachrichten zu validieren. Das gleiche Vorgehen wurde auch
für Nachrichten der Gangschaltung durchgeführt. Ein weiterer wichtiger Teil war die Echtzeitdekodierung der CAN-Bus Nachrichten,
damit auf Eingaben des Gashebels reagiert werden konnte. Dazu wurde das Programm \texttt{canReader.py} entwickelt. 
Die Rudersteuerung wurde nicht implementiert, da die Nachrichten nicht im Zeitrahmen dekodiert werden konnten.
\\
Im zweiten Test wurden manipulierte Nachrichten auf den CAN-Bus gesendet. In diesem Zustand wurden Nachrichten mit einer Zieldrehzahl von 650 Umdrehungen pro Minute gesendet. Dabei ist die 
Drehzahl zügig angestiegen, aber über 650 Umdrehungen pro Minute hinaus. Bei einer Drehzahl von 1500 Umdrehungen pro Minute wurde die
Nachrichtenübertragung gestoppt, um mögliche Schäden am Motor zu vermeiden. Dabei ist zu erwähnen, dass keine Fehlermeldung im Steuergerät 
aufgetreten ist. Das Verhalten des Motorsteuergeräts war nicht erwartbar nach den Nachrichten aus dem J1939-Standard. Daher
wurde von weiteren Tests abgesehen und es konnte auch nicht die Reaktion auf Gashebelbewegungen getestet werden. 

\subsection{Wie kann die Manipulation von Steuergeräten erschwert werden?}
Wie in dem Kapitel \ref{sec:schutzmassnahmen} genauer beschrieben, gibt es verschiedene Möglichkeiten, die Manipulation von Steuergeräten zu erschweren.
Die einfachste Methode ist dabei den physischen Zugang zu den Steuergeräten zu erschweren und möglicherweise für Unbefugte dauerhaft zu unterbinden.
Dazu kann die Verkabelung in einem geschlossenen Gehäuse untergebracht werden. 
Bei einem CAN-Bus ist eine weitere Maßnahme die Verwendung von Intrusion Detection Systemen (IDS).
Diese können den CAN-Bus überwachen und bei ungewöhnlichen Nachrichten Alarm schlagen. Diese arbeiten entweder Regelbasiert oder erkennen Anomalien in den Nachrichten.
\\
Eine Verschlüsselung von Nachrichten ist in der Theorie auch möglich, aber in der Praxis nicht umsetzbar. Das ist darauf zurückzuführen, dass die verschiedenen
Geräte von verschiedenen Herstellern stammen. Daher ist der Schlüsselaustausch nur schwer möglich. Eine weitere Möglichkeit ist die Verwendung von digitalen Signaturen.
Dies findet bei der Nutzung von einer Public Key Infrastructure (PKI) statt. Hier kann ein Gerät mit einem Zertifikat seine Identität beweisen. 
Eine solche PKI zu benutzen ist eine gerechtfertigte Maßnahme, wenn die Sicherheit der Kommunikation gewährleistet werden muss. \\

\subsection{Wie relevant sind die Ergebnisse für andere Schiffe?}
Der Mehrwert dieser Arbeit fokussiert sich nicht nur auf die Limanda, sondern auch auf andere Schiffe.
Aufgrund der weiten Verbreitung des CAN-Bus, ist davon auszugehen, dass auch viele andere Schiffe diesen nutzen.
Der CAN-Bus der Limanda hat keine speziellen Sicherheitsmaßnahmen, die über den Standard hinausgehen. 
Die Limanda wurde erst 2021 in Betrieb genommen und ist daher auf einem aktuellen Stand der Technik \cite{limanda}.
Es kann davon ausgegangen werden, dass auch andere Schiffe, die in den letzten Jahren in Betrieb genommen wurden, auf einem
ähnlichen Stand der Technik sind. Ein CAN-Bus in einem beliebigen Schiff ist anfällig für Manipulationen, wenn keine
Sicherheitsmaßnahmen, wie IDS oder PKI, implementiert sind. \\
In dem Fall der Limanda wurde nur der Motor mit einem CAN-Bus angesteuert. Es besteht jedoch die Möglichkeit, dass auch 
andere Systeme, wie die Rudersteuerung, mit einem CAN-Bus angesteuert werden. Wenn dies der Fall ist, dann 
ist die Gefahr einer erfolgreichen Fernsteuerung des Schiffes noch größer. Auf der Limanda ist das Ruder Mechanisch
mit dem Steuerrad verbunden. Daher ist die Gefahr hier geringer.
\\
NMEA-0183 Nachrichten werden im ASCII-Format übertrange. Dadurch können mindestens 
alle Informationen, die über NMEA-0183 übertragen werden, abgefangen werden. Das kann einen weiteren Angriffspunkt darstellen. \\
In dieser Arbeit konnten die Motoren nur manipuliert werden, wenn diese schon in Betrieb waren. Zusätzlich konnte
Manipulation durch das Ausschalten der Motoren momentan verhindert werden. Im Rahmen dieser Arbeit kann keine Aussage über 
die Möglichkeit der Manipulation von Motoren gemacht werden, die nicht in Betrieb sind. Es würde eine größere Gefahr
darstellen, wenn die Motoren von einem Angreifer ein- und ausgeschaltet werden könnten. \\

\section{Ausblick}
In dieser Arbeit wurde gezeigt, dass die Steuerung eines Schiffsmotor durch Manipulation von CAN-Bus Nachrichten möglich ist.
Allerdings konnte die Manipulation nicht auf vorhersehbare Weise durchgeführt werden. Die Motordrehzahl konnte nicht
vorhersehbar manipuliert werden. Eine Lösung dafür könnte in anderen Nachrichten von dem CAN-Bus liegen, welche nicht
dekodiert werden konnten. Daher könnte die Dekodierung von weiteren Nachrichten untersucht werden.
In den dekodierten Nachrichten konnten keine Informationen über die Gangschaltung gefunden werden. Daher ist anzunehmen,
dass diese Nachrichten auch nicht dekodiert werden konnten. Mit einer vollständigen Dekodierung der aller Nachrichten,
ist es wahrscheinlich, die Gangschaltung zu manipulieren. \\
Zusätzlich konnte nicht gezeigt werden, dass die Rudersteuerung manipuliert werden kann. Dies ist der begrenzten Zeit
und den fehlenden Informationen über die Rudersteuerung geschuldet. Es ist daher empfehlenswert, die Rudersteuerung
in einer weiteren Arbeit zu untersuchen. Dazu sollten die Nachrichten für die Rudersteuerung dekodiert werden. \\
In dieser Arbeit wurden Sicherheitsmaßnahmen vorgeschlagen, die nicht implementiert wurden. Diese müssen in einer weiteren
Arbeit auf Effektivität geprüft werden. Dazu können die Sicherheitsmaßnahmen implementiert und getestet werden.
Dabei kann auch die Reaktion auf Angriffe getestet werden. \\