\chapter{Abschließende Betrachtung}
\printmyminitoc{1}

\section{Ergebnis}
In dieser Arbeit sollte die Machbarkeit von Angriffen auf das Steuergerät eines Schiffes untersucht werden. 
Der Fokus lag dabei auf dem CAN-Bus für die Motorsteuerung und einer seriellen Verbindung für die Rudersteuerung.
Um Schwachstellen zu veranschaulichen, sollte der Motor sowie das Ruder mittels eines Spiele-Controllers gesteuert werden.
Im ersten Schritt wurde ein Konzept für die Steuerungslogik entwickelt, wie in \ref{sec:steuerungslogik} zu sehen.
Damit die physichen Eingaben genutzt werden konnten, mussten diese in logische Eingabewerte für ein Programm umgewandelt werden.
Das wurde in dem Programm \texttt{controllerInput.py} umgesetzt (\ref{sec:signalControllerSchiff}). 
Damit auf eigenen Systemen CAN-Nachrichten getestet werden konnten, wurde ein CAN-Bus aufgebaut und Testnachrichten gesendet.
Im nächsten Schritt wurden aufgezeichnete CAN-Bus Nachrichten mithilfe einer DBC-Datei dekodiert und anschließend analysiert. 
Dabei wurden auch Nachrichten für die Rudersteuerung aufgezeichnet, diese konnten jedoch nicht dekodiert werden.
Durch die Analyse der Nachrichten, konnten einzelne wichtige Nachrichten entdeckt werden. Durch die weitergehende Analyse
der entsprechenden DBC-Datei zu den bestimmten Nachrichten, konnten diese auch verstanden werden (\ref{sec:canBus}).
Damit konnte ein Programm entwickelt werden, welches die Eingaben des Spiele-Controllers erhält und in CAN-Bus Nachrichten umwandelt.
Dazu musste auch eine entsprechende Prüfsumme berechnet werden, um die Nachrichten zu validieren. Das gleiche Vorgehen wurde auch
für Nachrichten der Gangschaltung durchgeführt. Ein weiterer wichtiger Teil war die Echtzeitdekodierung der CAN-Bus Nachrichten,
damit auf Eingaben des Gashebels reagiert werden konnte. Dazu wurde das Programm \texttt{canReader.py} entwickelt. 
Die Rudersteuerung wurde nicht implementiert, da die Nachrichten nicht im Zeitrahmen dekodiert werden konnten.
\\
Im ersten Test wurden die Nachrichten auf dem CAN-Bus des Schiffes im normalen Betrieb mit dem Programm \texttt{canReader.py} dekodiert.
Dabei hat die Echtzeitdekodierung der Nachrichten funktioniert. Es konnten die wichtigen Nachrichten für die Motorsteuerung
erkannt werden. Allerdings konnten nicht alle Nachrichten dekodiert. Zusätzlich wurden keine Nachrichten für die Gangschaltung
dekodiert. Dabei ist wichtig zu erwähnen, dass diese Nachrichten nicht in der Standard DBC-Datei enthalten waren. Durch das Handbuch
des Motors konnte eine Nachricht nachgestellt werden. Eine solche Nachricht wurde aber nicht in der Kommunikation des Schiffes gefunden.
Im zweiten Test wurden manipulierte Nachrichten auf den CAN-Bus gesendet. In der Vorbereitung wurde die Gangschaltung von dem Kapitän in den neutralen
Gang geschaltet. In diesem Zustand wurden Nachrichten mit einer Zieldrehzahl von 650 Umdrehungen pro Minute gesendet. Dabei ist die 
Drehzahl zügig angestiegen, aber über 650 Umdrehungen pro Minute hinaus. Bei einer Drehzahl von 1500 Umdrehungen pro Minute wurde die
Nachrichtenübertragung gestoppt, um mögliche Schäden am Motor zu vermeiden. Dabei ist zu erwähnen, dass keine Fehlermeldung im Steuergerät 
aufgetreten ist. Das Verhalten des Motorsteuergeräts war nicht erwartbar nach den Nachrichten aus dem J1939-Standard. Daher
konnte auch nicht die Reaktion auf Gashebelbewegungen getestet werden. 
\subsection{1. Forschungsfrage}
\begin{itemize}
    \item Einzelne Erkenntnisse können auf andere Fahrzeuge übertragen werden
\end{itemize}
\subsection{2. Forschungsfrage}
\begin{itemize}
    \item Sicherheitsmaßnahmen wurden vorgeschlagen    
\end{itemize}

\section{Ausblick}
\begin{itemize}
    \item Motordrehzahl konnte bisher nicht auf vorhersehbare Weise manipuliert werden
    \item aufgezeichnete Nachrichten konnten nicht alle dekodiert werden
    \item darunter auch Nachrichten für die Gangschaltung
    \item Manipulation von Nachrichten für die Gangschaltung
    \item Rudersteuerung ist nicht implementiert
    \item Nachrichten dafür nicht ganz verstanden
    \item tatsächliche Implementierung von vorgeschlagenen Sicherheitsmaßnahmen
    \item Prüfen auf die Effektivität der Sicherheitsmaßnahmen
\end{itemize}