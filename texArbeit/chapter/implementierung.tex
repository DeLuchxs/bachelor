\chapter{Implementierung}
\printmyminitoc{1}

\section{Verbindung Rogue Device - Controller}
\begin{itemize}
    \item benutzt wird Python 3.12.8, da einfache Syntax und gute Bibliotheken
    \item Bibliothek: Pygame für die Controllereingabe (https://www.pygame.org/docs/(abgerufen am 08.12.2024))
    \item Pygame braucht ein GUI, andersfalls wird es Fehler ausgeben
    \item Beispielscode für Pygame: \href{https://github.com/kevinmcaleer/xbox_controller}{github} (abgerufen am 08.12.2024)
\end{itemize}


Benutzte Hardware, Protokolle, Libraries

\section{Übersetzung Signale Controller - Schiff}
Welches Dateiformat wird für Controllersignale benutzt?
Wie werden diese effizient genug in Motorsignale übersetzt?
Kann ich einfach originale Steuerungssignale unterdrücken?

Es gibt zuerst ein Programm, welches die Signale des Controllers erhält und in passende Variablen in Python interpretiert. Diese Variablen müssen
dann in Nachrichten für den Can-Bus umgewandelt werden. Hierfür gibt es ein weiteres Programm. Damit die beiden Programme miteinander kommunizieren
können, wird Inter-Process-Communication (IPC) benutzt. Als Methode werden hierbei Pipes benutzt. Diese sind einfach zu implementieren und haben
eine automatische Synchronisierung zwischen den Prozessen. Das bedeutet, dass die Prozesse nicht aufeinander warten müssen, sondern einfach
weiterarbeiten können. Es wird durch den Puffer der Pipe sichergestellt. Wenn dieser voll ist, wird der schreibende Prozess angehalten, bis der
lesende Prozess den Puffer geleert hat. Dies ist ein einfaches und effizientes Verfahren, 
um die beiden Prozesse zu synchronisieren \cite{Venkataraman2015}. 

\begin{itemize}
    \item Anfangsbeispiel: \href{https://www.geeksforgeeks.org/inter-process-communication-ipc/}{geeksforgeeks} (abgerufen am 18.12.2024)
    \item Python: \href{https://thelinuxcode.com/python-pipe-example/}{thelinuxcode} (abgerufen am 19.12.2024)
\end{itemize}

\section{Eingabe-Interface}
Wie wird die Rückmeldung tatsächlich aussehen?