\chapter{Grundlagen}

\printmyminitoc{1}

\section{CanBus}
Wie wird ein CanBus verdrahtet und wie kommunizieren die Geräte?
Was ist ein CanBus:
\begin{itemize}
    \item Technologie für serielle Netzwerke
    \item 1983 von Bosch für die Autoindustrie entwickelt
    \item Ist ein zweiadriger Bus, halbduplex
    \item konventionellen seriellen Technologien überlegen in Funktionalität und Zuverlässigkeit
    \item Kosteneffiizienter
    \item Entwickelt für Echtzeitanwendungen mit 1Mbit/s Baudrate
    \item Verwendung mittlwerweile allen möglichen Fahrzeugen, auch maritimer Bereich und Luftfahrt
    \item Medizinische Geräte, Industrieanlagen, Gebäudeautomation
\end{itemize}

\cite[Seiten 2-10]{Voss2008}
\\
Aufbau:
Alle Knoten sind mit zwei Drähten verbunden und sind gleichberechtigt.
\cite[Seite 132]{Voss2008}
\\
Die Nachrichten werden nach Broadcasting-Prinzip übertragen. Jede Nachricht wird von allen Knoten empfangen, 
aber nur die Knoten, die die Nachricht benötigen, verarbeiten sie. Diese werden aber nicht bestätigt,
da dies zu einer größeren Last auf dem Bus führen würde. Bei einer fehlerhaften Nachricht reagieren
die Knoten mit einer Fehlermeldung, die wieder der gesamte Bus empfängt.
\cite[Seite 80]{Voss2008}


\section{Raspberry Pi}
\begin{itemize}
    \item Ein Raspberry Pi ist ein Einplatinencomputer, der von der Raspberry Pi Foundation entwickelt wurde.
    \item mit diesem kann man viele Dinge machen, wie z.B. programmieren, Musik hören, Videos schauen, im Internet surfen, etc.
    \item Der Raspberry Pi hat viele Anschlüsse, wie z.B. USB, HDMI, Ethernet, Audio, etc.
    \item eignet sich gut um einfache Aufgaben zu erledigen, wie z.B. eine Webseite hosten, einen Fileserver betreiben, etc.
    \item 
\end{itemize}

\subsection{Raspberry Pi als Rogue Device}
Unter einem Rogue Device versteht man ein Gerät, welches sich unautorisiert und unauffällig in ein Netzwerk einbindet.
Dies kann ein Raspberry Pi sein, der sich in ein Netzwerk einbindet und Daten abgreift oder manipuliert. Hierüber können 
sich Angreifer Zugriff auf das Netzwerk verschaffen. 
\begin{itemize}
    \item eigener Code kann auf Rogue Device laufen
    \item kann von Angreifern von außen gesteuert werden
    \item kann auch genuzt werden um Informationen über das Netzwerk zu sammeln
\end{itemize}

\section{State of the Art}
\begin{itemize}
    \item Was gibt es schon für Lösungen?
    \item Was ist der aktuelle Stand der Technik?
    \item Was ist der aktuelle Stand der Forschung?
\end{itemize}
\subsection{Anbindung von Raspberry Pi an CanBus}
\begin{itemize}
    \item Raspbian OS hat seit 05.05.2015 eingebundenen Support für den Mikrochip MCP251x
\end{itemize}
\cite{Salunkhe2016}
\begin{itemize}
    \item PiCan2 ist ein CanBus Board für den Raspberry Pi
    \item HAT (Hardware Attached on Top) Standard
    \item erlaubt dem Raspberry Pi mit dem CanBus zu kommunizieren mit einer Geschwindigkeit von bis zu 1Mbit/s
\end{itemize}
\cite{Pant2019}
\subsection{Übersetzung von CanBus Nachrichten}
\begin{itemize}
    \item J1939 (DBC-Dateien) (https://docs.fileformat.com/de/database/dbc/)
    \item NMEA 0183 
    \item NMEA 2000 $\rightarrow$ CanBoat (https://github.com/canboat/canboat)
\end{itemize}