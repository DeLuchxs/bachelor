\chapter{Grundlagen}

\printmyminitoc{1}

\section{CanBus}
Wie wird ein CanBus verdrahtet und wie kommunizieren die Geräte?
Was ist ein CanBus:
\begin{itemize}
    \item Technologie für serielle Netzwerke
    \item 1983 von Bosch für die Autoindustrie entwickelt
    \item Ist ein zweiadriger Bus, halbduplex
    \item konventionellen seriellen Technologien überlegen in Funktionalität und Zuverlässigkeit
    \item Kosteneffiizienter
    \item Entwickelt für Echtzeitanwendungen mit 1Mbit/s Baudrate
    \item Verwendung mittlwerweile allen möglichen Fahrzeugen, auch maritimer Bereich und Luftfahrt
    \item Medizinische Geräte, Industrieanlagen, Gebäudeautomation
\end{itemize}

\cite[Seiten 2-10]{voss2008comprehensible}
\\
Aufbau:
Alle Knoten sind mit zwei Drähten verbunden und sind gleichberechtigt.
\cite[Seite 132]{voss2008comprehensible}
\\
Die Nachrichten werden nach Broadcasting-Prinzip übertragen. Jede Nachricht wird von allen Knoten empfangen, 
aber nur die Knoten, die die Nachricht benötigen, verarbeiten sie. Diese werden aber nicht bestätigt,
da dies zu einer größeren Last auf dem Bus führen würde. Bei einer fehlerhaften Nachricht reagieren
die Knoten mit einer Fehlermeldung, die wieder der gesamte Bus empfängt.
\cite[Seite 80]{voss2008comprehensible}


\section{Raspberry Pi}

\subsection{Raspberry Pi als Rogue Device}
Was ist ein Rogue Device und wie wird der RasPi als solches eingesetzt