\documentclass[usenames, dvipsnames, aspectratio=75]{beamer}
\usepackage[utf8]{inputenc}
\usepackage[T1]{fontenc}
\usepackage[ngerman]{babel}
\usepackage[defaultfam, regular]{montserrat}
\usepackage[german=guillemets]{csquotes}
\usepackage{hyperref}
    \hypersetup{colorlinks=true, citecolor=purple, filecolor=black, linkcolor=TealBlue, urlcolor=blue}
\usepackage{latexsym,multicol,booktabs,miama}
\usepackage{setspace}
\usepackage{xcolor}
\usepackage{tikz}
\usetikzlibrary{shapes.geometric, arrows}
\tikzstyle{task} = [rectangle, rounded corners, text centered, draw=black,
    minimum width=4cm, minimum height=1.5cm]
\usepackage{amssymb,amsfonts,amsmath,amsthm,mathrsfs,mathptmx} 
\usepackage{graphicx,pstricks,listings,stackengine}
\usepackage[figurename=Fig., tablename=Tab., font=footnotesize]{caption}
\usepackage[backend=biber,bibencoding=utf8,urldateusetime=true,style=alphabetic,sorting=nyt,natbib=true]{biblatex}
\usepackage{tcolorbox} % For creating colored boxes
\addbibresource{assets/refs.bib}

\setcounter{biburllcpenalty}{7500}
\setcounter{biburlucpenalty}{8000}

\setbeamertemplate{bibliography item}{\insertbiblabel}

\usepackage{todonotes}
\newcommand{\newtodo}[1]{\vspace{.75em}\todo[inline,color=yellow]{#1}}
\setlength {\marginparwidth }{2cm}

\author{Jakob Engelbert Tomahogh}

\title[]{Sicherheitsanalyse durch Entwicklung eines Rogue Device zur Echtzeitmanipulation maritimer Steuerungssysteme}
\institute[Universität Rostock]{
\inst{}
\footnotesize{Betreuer: \textit{M.Sc. Marvin Davieds}} \vspace*{.25em} \\
\footnotesize{Zweitgutachter: \textit{Prof. Dr. rer. nat. Clemens H. Cap}}
}
\date{14.02.2025}

\usepackage{assets/EXTRAS}

\def\cmd#1{\texttt{\color[RGB]{0, 0, 139}\footnotesize $\backslash$#1}}
\def\env#1{\texttt{\color[RGB]{0, 0, 139}\footnotesize #1}}

\lstdefinestyle{texstyle}{
    language=[LaTeX]TeX,
    basicstyle=\ttfamily\footnotesize,
    keywordstyle={\bfseries\color[RGB]{0, 0, 180}},
    keywordstyle=[2]{\slshape\color[RGB]{225, 140, 30}},
    morekeywords=[2]{,itemize, enumerate, equation, table, tabular},
    stringstyle=\color[RGB]{50, 50, 50},
    numbers=left,
    numberstyle=\footnotesize\color{gray},
    rulesepcolor=\color{red!20!green!20!blue!20},
    frame=shadowbox
}

\begin{document}

\begin{frame}
    \vspace*{.25em}
    \begin{figure}[htpb]
        \centering
        \includegraphics[width=0.55\linewidth]{assets/logo_uni_rostock.jpg}
    \end{figure}
    \vspace*{-1.5em}
    \titlepage
\end{frame}

\section{Einführung}

\subsection{Motivation}

\begin{frame}{Motivation}
    \begin{itemize}
        \item Sicherheit wurde in maritimen Systemen vernachlässigt
        \item Kommunikationssysteme sind anfällig für Angriffe
        \item Angriff auf Steuerungssysteme könnte katastrophale Folgen haben
        \item Ziel ist: Bewusstsein für Sicherheit schaffen und die Einfachheit eines Angriffs zeigen
    \end{itemize}
\end{frame}


% ---------------------------------------------------------------------------

\section{Grundlagen}

\subsection{CAN-Bus}
\begin{frame}
    \begin{itemize}
        \item serielle Netzwerktechnologie, bei dem mehrere Geräte miteinander kommunizieren können
        \item ermöglichicht effiziente Kommunikation zwischen Steuergeräten
        \item alle Geräte sind gleichberechtigt
    \end{itemize}
\end{frame}

\subsection{bekannte Sicherheitslücken}
\begin{frame}
    \begin{itemize}
        \item Kommunikation auf dem CAN-Bus ist unverschlüsselt
        \item Ein zusätzliches Gerät kann ohne Probleme in das Netzwerk eingebunden werden
        \item Nachrichten mit niedrigerer ID haben höhere Priorität
    \end{itemize}
\end{frame}
% ---------------------------------------------------------------------------

\section{Konzept}
\begin{frame}
    \begin{itemize}
        \item Entwicklung eines Rogue Devices
        \item Anbindung an das Schiff über den CAN-Bus 
        \item Eingabe der Befehle über einen Xbox Series X Controller
        \item Übersetzung auf dem Rogue Device
    \end{itemize}
\end{frame}

\section{derzeitiger Stand}
\begin{frame}
    \begin{itemize}
        \item Anbindung des Raspberry Pi an den CAN-Bus
        \item Steuerung des Schiffes über den Controller
    \end{itemize}
\end{frame}

\section{Ausblick}
\begin{frame}
    \begin{itemize}
        \item Implementierung der Übersetzung der Befehle
        \item Testen der Steuerung
        \item Analyse der Auswirkungen
    \end{itemize}
\end{frame}

% ---------------------------------------------------------------------------

\begin{frame}[allowframebreaks]{References}
    \printbibliography[heading=none]
    \begin{itemize}
        \item alle Quellen der Arbeit oder nur die der Präsentation?
    \end{itemize}
\end{frame}

\end{document} 

